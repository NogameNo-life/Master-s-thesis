% !TeX encoding = UTF-8 Unicode
% !TeX root = main.tex
% !TeX TS-program = pdflatex
%% (При смене движка необходимо удалить вспомогательные файлы *.aux *.brf *.log *.out *.synctex.gz *.toc)

\documentclass{thesisby}
%\usepackage{etoolbox,ifxetex,ifluatex}
\usepackage[unicode,hypertexnames=false]{hyperref}
\usepackage[
    language = auto,        % Получение языка из babel.
    autolang = other,       % Многоязычная библиография.
    defernumbers = true,    % Раздельная нумерация.
    style = gost-numeric,
    maxnames = 10,
    movenames = false,
    sorting = ynt
]{biblatex} % To load multiple bib files.

%%% Проверка используемого TeX-движка %%%
\ifboolexpr{bool{xetex} or bool{luatex}}{%
  \usepackage{fontspec}
  \PassOptionsToPackage{no-math}{fontspec}     % https://tex.stackexchange.com/a/26295/104425
  \usepackage{polyglossia}%[2014/05/21]        % Поддержка многоязычности

  % fonts and languages
  \defaultfontfeatures{Ligatures=TeX,Mapping=tex-text}

  \setmainlanguage[babelshorthands = true]{russian}
  \setotherlanguage{english}

  \setmainfont{Times New Roman}
  \setmonofont{Courier New}
  \setsansfont{Arial}

  \newfontfamily\cyrillicfont[Script=Cyrillic]{Times New Roman}
  \newfontfamily\cyrillicfontsf[Script=Cyrillic]{Arial}
  \newfontfamily\cyrillicfonttt[Script=Cyrillic]{Courier New}

  \newfontfamily\englishfont{Times New Roman}
  \newfontfamily\englishfontsf{Arial}
  \newfontfamily\englishfonttt{Courier New}

  \renewcommand{\UrlFont}{\small\rmfamily\tt}
}{%
  \usepackage[T1,T2A]{fontenc}
  \usepackage[utf8]{inputenc}
  \usepackage[english, russian]{babel}
  \usepackage{csquotes}
  \IfFileExists{pscyr.sty}{\usepackage{pscyr}}{}  % Подключение pscyr
}

% Для борьбы с переполнениями за счет разреженных слов в абзаце
\emergencystretch=25pt

\usepackage{enumitem}

% Библиографический список в соответствии с ГОСТ.
\makeatletter
\ltx@iffilelater{biblatex-gost.def}{2017/02/01}%
{\toggletrue{bbx:gostbibliography}%
    \renewcommand*{\revsdnamepunct}{\addcomma}}{}
\makeatother

% Общий список.
\addbibresource{bibtex_base.bib}

% Список публикаций соискателя.
\addbibresource{bibtex_my.bib}
\DeclareSourcemap{
    \maps[datatype=bibtex, overwrite]{
        \map{
            \perdatasource{bibtex_my.bib}
            \step[fieldset=KEYWORDS, fieldvalue=idzm, append]
        }
    }
}

% Счётчики.
\usepackage[figure,table]{totalcount}   % Счётчик рисунков и таблиц.
\usepackage{totcount}                   % Пакет создания счётчиков на основе последнего номера подсчитываемого элемента (может требовать дважды компилировать документ).
\AtEveryBibitem{\stepcounter{citenum}\stepcounter{citenum_my}}

\usepackage{totpages}
\usepackage[abspage, user, lastpage]{zref}

\usepackage{microtype}

%for lists
\usepackage[ampersand]{easylist}
\ListProperties(Hide=100, Hang=false, Margin=0mm, Indent1=10.5mm, Indent2=15mm, Style*=-- ,
Style2*=$\bullet$ ,Style3*=$\circ$ ,Style4*=\tiny$\blacksquare$ )

\newenvironment{easylistNum}{
    \begin{easylist}
        \ListProperties(Hide1=0, Hang=false, Margin=0mm, Indent1=10.5mm, Indent2=15mm, Start1=1, Style*=, FinalMark={)})}
        {\ListProperties(Hide=100, Hang=false, Margin=0mm, Indent1=10.5mm, Indent2=15mm, Style*=-- , Style2*=$\bullet$ ,Style3*=$\circ$ ,Style4*=\tiny$\blacksquare$ )
    \end{easylist}}

\usepackage{amsmath, amssymb, amsfonts}
\usepackage{mathtools} % Use \rcases
\usepackage{longtable, array}
\usepackage{graphicx, epsfig}

\usepackage{algorithm}        % Для вставки псевдокода
\usepackage{algpseudocode}    % Для вставки псевдокода

% Русская традиция начертания греческих букв
\usepackage{upgreek} % Прямые греческие ради русской традиции (в формулах записывается \alpha как \upalpha и т.д.)

\usepackage{siunitx} % For Celsium sign only

\begin{document}

% Регистрируем счётчики в системе "totcount".
\regtotcounter{totalcount@table}
\regtotcounter{totalcount@figure}
\regtotcounter{TotPages}
\regtotcounter{textpages}

% Вычисление страниц только с текстом.
\makeatletter
\def\textpages{\number\numexpr\zref@extract{lastpagetocount}{abspage}-\zref@extract{firstpagetocount}{abspage}+1\relax}
\makeatother

\newtotcounter{textpages}
\setcounter{textpages}{\textpages}

% Вычисление количества источников.
\newtotcounter{citenum}
\newtotcounter{citenum_my}

\hypersetup{
    pdftitle = {НЕЙРОСЕТЕВЫЕ МЕТОДЫ АСУТП},
    pdfauthor = {Ситковец Яна Сергеевна},
    pdfsubject = {Диссертация},
    pdfkeywords = {ТеХ, диссертация}
}

 % !TEX encoding = UTF-8 Unicode
\begin{titlepage}

    \begin{center} \bfseries
        % Национальная академия наук Беларуси\\
        \bigskip
        % {Учреждение образования}
        \medskip

        {БЕЛОРУССКИЙ ГОСУДАРСТВЕННЫЙ УНИВЕРСИТЕТ}
    \end{center}
    \vspace{1cm}

    \noindent УДК 004.032.26 \\
    \vspace{1cm}

    \begin{center}
        {Ситковец \\ Яна Сергеевна}\\
        \vspace{1cm}

        {\bfseries НЕЙРОСЕТЕВЫЕ МЕТОДЫ АСУТП}\\
        \vspace{2cm}
        Диссертация потенциально будущего\\
        магистра технических наук\\
        \bigskip

        по специальности 05.13.17 -- Теоретические основы информатики
    \end{center}
    \vspace{3cm}

    \begin{tabbing}
        \hspace{8cm} \= \kill \>
        Научный руководитель \+ \\
        будущий кандидат технических наук, будущий доцент\\
        Иванюк Д. С.
    \end{tabbing}


    \ifdefined\dissertationversion
        \vspace{3cm}
        \begin{center}
            \bfseries v\dissertationversion
        \end{center}
        \vspace{3cm}
    \else
        \vspace{7cm}
    \fi

    \begin{center}
        \bfseries Брест 2024
    \end{center}

\end{titlepage}



\end{document}