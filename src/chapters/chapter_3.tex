\chapter{ ГЛАВА 3. Реализация модели в Timefold Solver}
\label{ch:chapter3}

\section{ Выбор инструмента}

Для реализации задачи оптимизации выбрана библиотека Timefold Solver — современный инструмент для решения задач планирования и расписания с использованием эвристик и метаэвристик. Timefold позволяет описывать модель задачи через аннотированные Java-классы и автоматически подбирать эффективную стратегию поиска решения.

\subsection{Модель планирования}

Модель включает следующие компоненты: \

Класс Job (фасуемое задание)

Аннотирован как @PlanningEntity. Содержит:

\begin{itemize}
    \item String id, String name — идентификаторы;
    \item Product product — тип продукта (с глазурью);
    \item int quantity — количество сырков;
    \item Line line — назначаемая линия;
    \item Duration getDuration() — длительность, вычисляемая динамически:
\end{itemize}

Класс Line (фасовочная линия). Аннотирован как @PlanningEntity. Хранит список заданий:

@PlanningListVariable \
private List<Job> jobs; \

Каждая линия должна обрабатывать задания последовательно с учётом длительности и времени мойки между несовместимыми продуктами. \

\subsection{Ограничения и ConstraintProvider}

Ограничения описаны в классе FoodPackagingConstraintProvider. Примеры:

Жёсткое ограничение: несовместимость линии и типа продукта \

Конфигурация решателя cоздаётся через SolverConfig: \

SolverFactory<PackagingSchedule> solverFactory = SolverFactory.create(new SolverConfig()
    .withSolutionClass(PackagingSchedule.class)
    .withEntityClasses(Job.class, Line.class)
    .withConstraintProviderClass(FoodPackagingConstraintProvider.class)
    .withTerminationSpentLimit(Duration.ofMinutes(5)));

\subsection{Экспорт результатов}

Решение сохраняется в JSON-файл, содержащий:

\begin{itemize}
    \item список заданий с назначенными линиями;
    \item длительность фасовки;
    \item ID, тип продукта и глазури;
    \item дата производственного заказа.
\end{itemize}

кспорт реализован в классе DataExporter.