\chapter*{Заключение}
\addcontentsline{toc}{chapter}{Заключение}

В рамках выполненного исследования была рассмотрена и реализована комплексная система, направленная на оптимизацию работы фасовочных линий с использованием современного инструментария в области искусственного интеллекта и дискретной оптимизации. Основное внимание было сосредоточено на применении Timefold Solver — одного из наиболее эффективных и гибких решений для задач производственного планирования, а также на разработке вспомогательных компонентов, облегчающих генерацию ограничений при помощи моделей машинного перевода и генеративных нейросетей.

Проблематика, затронутая в данной работе, имеет непосредственное практическое значение для предприятий, задействованных в упаковке и фасовке продукции. Как показано в теоретической части исследования, планирование загрузки фасовочных линий представляет собой сложную задачу с множеством жёстких и мягких ограничений, которые охватывают различные аспекты — от технологических регламентов до требований по смене аллергенов и минимизации времени на мойку оборудования. Учитывая высокий уровень сложности и большое количество переменных, традиционные подходы, основанные на ручной настройке или на шаблонных ERP-решениях, нередко оказываются недостаточно эффективными, особенно при масштабировании задач.

В ходе исследования была сформализована математическая постановка задачи, учитывающая специфические ограничения, характерные для фасовочного производства. Были разработаны и реализованы модели предметной области, отражающие реальные производственные процессы, включая описание продукции, технологических линий, типов глазури и аллергенных характеристик. Все данные были интегрированы в планировщик Timefold, где была построена система ограничений с использованием подхода Constraint Streams, позволяющего эффективно выражать сложные логические зависимости в виде потоков ограничений на языке Java.

Однако ключевым вкладом данной работы стало внедрение дополнительного уровня автоматизации — механизма генерации ограничений на основе естественного языка. Использование нейросетевого переводчика и генератора Java-кода позволило реализовать интерфейс, в котором пользователь может описывать производственные требования и бизнес-правила на русском языке, не прибегая к ручному программированию. Такой подход существенно снижает технический порог входа, упрощает разработку, позволяет быстрее адаптировать модель под изменяющиеся условия и стимулирует активное участие специалистов-предметников в процессе построения модели.

В работе также была реализована архитектура системы, состоящей из трёх контейнеризованных сервисов: FastAPI-переводчика (на основе модели Helsinki-NLP), генератора Java-кода (на основе модели CodeGen), а также веб-интерфейса, построенного на базе Next.js. Все компоненты были упакованы в Docker-контейнеры, что обеспечивает простоту развертывания, масштабируемость и возможность локального использования на сервере с GPU. Такой подход позволяет предприятиям, обеспокоенным конфиденциальностью данных или ограничениями в доступе к интернет-сервисам, внедрять интеллектуальные инструменты внутри корпоративной инфраструктуры.

Результаты тестирования системы показали её жизнеспособность и практическую применимость. Сформированные ограничения, сгенерированные на основе описаний на естественном языке, успешно интегрировались в планировочную модель Timefold и позволили получить корректные решения в рамках заданных сценариев. Несмотря на необходимость финальной валидации сгенерированного кода и возможные ошибки перевода при недостаточно чётких формулировках, предложенный подход значительно ускоряет процесс создания модели и снижает трудозатраты на этапе её сопровождения.

Следует также отметить выявленные ограничения. Так, нейросетевые модели, работающие без учёта контекста или обучающего корпуса, специализированного под задачи дискретной оптимизации, могут генерировать код с синтаксическими или логическими ошибками. Кроме того, успешное функционирование всей архитектуры требует определённых вычислительных ресурсов, включая наличие GPU и корректную настройку сред выполнения. Однако, как показано в работе, данные ограничения являются технически преодолимыми и не нивелируют общей пользы от внедрения интеллектуальных вспомогательных инструментов в процессы планирования.

Таким образом, можно сделать вывод, что разработанная система представляет собой перспективное направление в области интеллектуализации производственного планирования. Она позволяет объединить преимущества строгой формальной постановки задач и гибкости, присущей современным генеративным моделям. Дальнейшее развитие данного направления может включать:
\begin{itemize}
    \item дообучение моделей генерации кода на корпусе Java-ограничений, специфичных для Timefold;
    \item реализацию обратной связи и объясняемости результатов (explainability) для повышения доверия к сгенерированным ограничениям;
    \item интеграцию пользовательского интерфейса с визуальной проверкой и отладкой расписаний;
    \item расширение системы на другие типы производственных задач — например, планирование смен, логистику или складскую обработку.
\end{itemize}

В заключение, результаты данной магистерской диссертации демонстрируют, что интеграция нейросетевых моделей в цикл разработки оптимизационных решений может значительно повысить эффективность работы специалистов и ускорить внедрение цифровых инструментов на предприятиях. Предложенный подход имеет потенциал для тиражирования и адаптации в других отраслях, что делает его ценным вкладом в развитие прикладного искусственного интеллекта в промышленности.
