\chapter*{ЗАКЛЮЧЕНИЕ}
\addcontentsline{toc}{chapter}{ЗАКЛЮЧЕНИЕ}

В рамках выполненного исследования была рассмотрена и реализована комплексная система, направленная на оптимизацию работы фасовочных линий с использованием современного инструментария в области искусственного интеллекта и дискретной оптимизации. Основное внимание было сосредоточено на применении Timefold Solver — одного из наиболее эффективных и гибких решений для задач производственного планирования, а также на разработке вспомогательных компонентов, облегчающих генерацию ограничений при помощи моделей машинного перевода и генеративных нейросетей.

Проблематика, затронутая в данной работе, имеет непосредственное практическое значение для предприятий, задействованных в упаковке и фасовке продукции. Как показано в теоретической части исследования, планирование загрузки фасовочных линий представляет собой сложную задачу с множеством жёстких и мягких ограничений, которые охватывают различные аспекты — от технологических регламентов до требований по смене аллергенов и минимизации времени на мойку оборудования. Учитывая высокий уровень сложности и большое количество переменных, традиционные подходы, основанные на ручной настройке или на шаблонных ERP-решениях, нередко оказываются недостаточно эффективными, особенно при масштабировании задач.

В ходе исследования была сформализована математическая постановка задачи, учитывающая специфические ограничения, характерные для фасовочного производства. Были разработаны и реализованы модели предметной области, отражающие реальные производственные процессы, включая описание продукции, технологических линий, типов глазури и аллергенных характеристик. Все данные были интегрированы в планировщик Timefold, где была построена система ограничений с использованием подхода Constraint Streams, позволяющего эффективно выражать сложные логические зависимости в виде потоков ограничений на языке Java.

В рамках данной работы было проведено исследование возможности автоматизации процесса генерации Java-кода ограничений на базе текстовых описаний, заданных на естественном языке. Целью экспериментов было не столько построение готового промышленного решения, сколько анализ применимости современных языковых моделей для задач проектирования оптимизационных моделей с использованием API Constraint Streams библиотеки Timefold Solver.

Проведённые тесты показали, что даже при использовании относительно компактной модели (\texttt{codegen-2B-multi}) возможно получение отдельных валидных элементов кода, включая конструкции \texttt{from()}, \texttt{filter()} и \texttt{penalize()}. Однако общая точность и соответствие сгенерированного кода требованиям предметной области оказались недостаточными для прямого включения результатов в планировочную систему без дополнительной ручной доработки.

Особое влияние на качество генерации оказал этап машинного перевода — первоначальный текст на русском языке часто преобразовывался с искажениями, в частности, ключевые термины, такие как «Constraint», «Constraint Streams», искажались до «Consign», «Constrat» и других некорректных форм. Эти ошибки негативно влияли на семантику и мешали языковой модели корректно распознать задачу. Таким образом, важно подчеркнуть, что многоэтапный процесс генерации — особенно при наличии промежуточного перевода — требует тщательной фильтрации и контроля качества на каждом этапе.

Также было выявлено, что архитектура генератора недостаточно гибка для обработки сложных структурных промптов, включающих множественные ограничения и контекстуальные зависимости между объектами. Модель не справлялась с генерацией классов, соответствующих интерфейсу \texttt{ConstraintProvider}, ошибалась в сигнатурах методов и игнорировала внутреннюю логику API Timefold Solver.

Тем не менее, даже частичные положительные результаты подтверждают, что нейросетевые модели могут быть полезны на ранних этапах проектирования ограничений — в частности, для генерации заготовок, шаблонов и базовой структуры методов. Это может ускорить процесс разработки, особенно в случае типовых ограничений и хорошо сформулированных описаний.

Разработанная архитектура системы с использованием Docker-контейнеров продемонстрировала стабильную работу и удобство в развёртывании, а также возможность локального использования без зависимости от внешних API. Такой подход может быть масштабирован и перенесён в другие сценарии промышленного планирования.

В дальнейшем, для повышения эффективности генерации кода, целесообразно рассмотреть:

\begin{itemize}
\item замену этапа перевода на многоязычные модели, напрямую работающие с русским языком;
\item дообучение языковых моделей на корпусе существующих ограничений, специфичных для Timefold;
\item реализацию механизма обратной связи и валидации кода непосредственно в интерфейсе пользователя;
\item расширение набора промптов с учётом различных типов ограничений и сценариев использования.
\end{itemize}

Проведённое исследование продемонстрировало, что использование нейросетей в задачах генерации оптимизационного кода является перспективным направлением, однако требует дальнейших усилий в области дообучения моделей, повышения надёжности перевода и адаптации архитектур к спецификам предметной области.
