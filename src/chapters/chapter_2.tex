\chapter{Описание задачи планирования}
\label{ch:chapter2}

\section{Постановка задачи и описание модели}

В современном пищевом производстве задачам автоматизированного планирования уделяется особое внимание. Особенно это актуально в условиях высокой вариативности продукции, ограниченных ресурсах и необходимости строго соблюдать санитарные и технологические нормы. Целью данной работы является разработка интеллектуального алгоритма, обеспечивающего эффективное распределение заданий на упаковку глазированных сырков между производственными линиями с учетом различных производственных и технологических ограничений.

Процесс планирования включает в себя множество факторов, начиная от объёмов производственного заказа и заканчивая необходимостью проведения санитарных обработок после определённых видов продукции. В условиях ограниченного количества линий и высокой загрузки предприятия особенно важна оптимизация:

\begin{enumerate}
	\item Суммарного времени переналадок, включая санитарные мойки.
	\item Простоя оборудования, не участвующего в производстве.
	\item Соблюдения сроков выработки приоритетных партий.
	\item Ресурсов — как человеческих, так и технологических (например, варочные мельницы для глазури).
\end{enumerate}

Таким образом, планировщик должен выдавать расписание, минимизирующее суммарные издержки и обеспечивающее выполнение производственного плана.

На входе системе планирования передаётся разбивка производственного заказа на партии (jobs). Каждая партия представляет собой определённое количество сырков конкретного вида. На выходе алгоритм выдает расписание: какие партии, в какой последовательности, на какой линии, и в какое время должны быть упакованы.

\section{Исходные данные о работе производства}

Предприятие располагает шестью автоматизированными фасовочными линиями. По конструктивным особенностям и скорости работы они делятся на два класса:
\begin{itemize}
    \item Линии 1, 2, 3 — модели GSL6, с меньшей производительностью.
    \item Линии 4, 5, 6 — модели GSL8, с повышенной скоростью и наличием дополнительного оборудования, например, фруктопитателя.
\end{itemize}

Каждая линия может обслуживать ограниченное множество типов продукции. Это связано с тем, что определённые типы сырков требуют специфических компонентов, насадок или обработки (например, глазурь с аллергенами или фруктовая начинка).

Типы продукции представлены в таблице \ref{table:ProductsType}. Разделение по типам играет ключевую роль при назначении заданий на линии, так как от этого зависит потребность в мойке, переналадке и допустимости производства.

\begin{table}[h]
\centering
\caption{Типы продукции}
\begin{tabularx}{\textwidth}{|l|X|}
\hline
\textbf{Тип продукта} & \textbf{Комментарий} \\
\hline
Классика                   & Основной тип, включает множество начинок и глазурей (флоупак, 40 г). \\
\hline
Плюш                       & Производится только на первой линии (фольга, 45 г). \\
\hline
Стержень (Топ, Творобушки) & Требует переналадок, связанных с мойкой фруктопитателя (флоупак, 40 г). \\
\hline
Кактус                     & Требует переналадок до и после, длительностью около 3--4 часов (флоупак, 40 г). \\
\hline
\end{tabularx}
\label{table:ProductsType}
\end{table}

\section{Ограничения и особенности}

Реальное производство сопровождается множеством технологических и санитарных ограничений. Их необходимо учитывать при построении расписания. Ниже перечислены основные ограничения:

\begin{itemize}
  \item \textbf{Ограничение по объёму:} общее количество сырков, которое может быть произведено за сутки, ограничено 44 тоннами.
  \item \textbf{Очередность:} смена продукта с одной начинкой на другую может требовать частичной или полной мойки оборудования.
  \item \textbf{Глазурь:} варка глазури производится на отдельных мельницах, которых всего три. Это накладывает ограничения по количеству и времени её подготовки.
  \item \textbf{Ежесуточная мойка:} каждая линия должна быть подвергнута полной мойке не реже одного раза в сутки.
  \item \textbf{Аллергены:} продукция, содержащая аллергены (арахис, яйцо и др.), требует обязательной мойки до и после производства.
  \item \textbf{Особые случаи:} например, "Кактус" требует длительной подготовки и переналадки (до 4 часов), что делает его планирование особенно чувствительным.
\end{itemize}

\section{Типы переналадок и длительность моек}

Серьёзным фактором, влияющим на производительность, являются смены продукции и, как следствие, санитарные мероприятия. В таблице \ref{table:CleaningTimeBetweenProducts} приведены основные типы переналадок и соответствующее время, затрачиваемое на них.

\begin{table}[h]
\centering
\caption{Типы переналадок}
\begin{tabularx}{\textwidth}{|l|X|}
\hline
\textbf{Тип переналадки} & \textbf{Мойка} \\
\hline
Стержень-классика                                                    & 2.5 часа. \\
\hline
Стержень смена наполнителя                                           & 50 минут. \\
\hline
Смена глазури с мойкой (аллергены)                                   & 1.5 часа. \\
\hline
Смена глазури без мойки                                              & 40 минут. \\
\hline
Без смены глазури                                                    & 20 минут. \\
\hline
Плюш-классика                                                        &  2,5-3 часа.\\
\hline
Стержень с начинкой на стержень без начинки                          & 40 минут. \\
\hline
Смена пленки                                                         & 5-10 минут. \\
\hline
\end{tabularx}
\label{table:CleaningTimeBetweenProducts}
\end{table}

\section{Формализация задачи}

Для эффективного решения задачи используется объектно-ориентированная модель. Она описывает основные сущности:

\begin{itemize}
    \item \textbf{Product (Продукт):} содержит информацию о типе сырка и глазури.
    \item \textbf{Job (Задача):} партия продукции определённого типа и объёма.
    \item \textbf{Line (Линия):} упаковочная линия, на которую назначаются задания.
\end{itemize}

Построение расписания требует соблюдения:

\textbf{Жёстких ограничений:}
\begin{itemize}
    \item Задание может быть назначено только на допустимую линию.
    \item Временные интервалы заданий не должны пересекаться.
    \item Обязательно соблюдение санитарных требований (мойки).
\end{itemize}

\textbf{Мягких ограничений:}
\begin{enumerate}
    \item Предпочтение заданиям с одинаковой глазурью/типом подряд.
    \item Снижение количества разрывов в заказе (желательно не разбивать).
    \item Балансировка нагрузки между линиями.
\end{enumerate}

\subsection{Математическая модель}

Пусть:

\begin{itemize}
    \item $J = \{j_1, j_2, ..., j_n\}$ — множество заданий;
    \item $L = \{l_1, l_2, ..., l_m\}$ — множество линий;
    \item $q_i$ — объём задания $j_i$;
    \item $s_{k,t}$ — скорость линии $l_k$ при упаковке типа $t$;
    \item $d_{i,k} = \frac{q_i}{s_{k,t}}$ — продолжительность задания $j_i$ на линии $l_k$;
    \item $c_{i,j}$ — время переналадки между $j_i$ и $j_j$;
    \item $x_{i,k} \in \{0,1\}$ — 1, если $j_i$ назначено на $l_k$;
    \item $start_i$ — момент начала выполнения $j_i$;
    \item $end_i = start_i + d_{i,k}$ — момент окончания.
\end{itemize}

Целевая функция:

\begin{equation}
    \min \left( w_1 \cdot \sum_{i,j} c_{i,j} + w_2 \cdot \text{makespan} + w_3 \cdot \sum_{\text{split orders}} penalty \right)
    \label{eq:target_function}
\end{equation}

\noindent где:
\begin{itemize}
    \item $w_1, w_2, w_3$ — веса критериев;
    \item $makespan = \max \limits_i end_i$ — длительность расписания;
    \item penalty — штраф за разбиение одного заказа на несколько непоследовательных заданий.
\end{itemize}

\subsection{Производительность линий}

Производительность зависит как от линии, так и от типа продукта. Пример значений приведён в таблице:

\begin{table}[h]
\centering
\caption{Производительность линий}
\begin{tabularx}{\textwidth}{|l|c|X|}
\hline
\textbf{№ Линии} & \textbf{Кол/мин} & \textbf{Тип продукта} \\
\hline
1,2,3 (GSL6) & 192–198 & Классика \\
\hline
6 (GSL8)     & 240     & Классика \\
\hline
1 (GSL6)     & 164     & Плюш \\
\hline
1,2,3 (GSL6) & 186–188 & Кактус \\
\hline
4,5,6 (GSL8) & 198     & Стержень \\
\hline
\end{tabularx}
\end{table}

\subsection{Заключение}

Таким образом, формализованная задача планирования упаковки глазированных сырков представляет собой типичную, но в то же время сложную задачу дискретной оптимизации, характерную для реального производственного процесса. Её особенностью является наличие множества конфликтующих ограничений — как жёстких (непересекающиеся интервалы заданий, совместимость линии и продукта, санитарные требования), так и мягких (минимизация времени мойки, предпочтение непрерывной упаковки одного вида продукции, балансировка загрузки).

Кроме того, задача обременена реальными производственными условиями: ограничениями по ресурсам (ограниченное число линий, варочных мельниц), сложной логикой переналадок, непостоянной скоростью упаковки в зависимости от комбинации линия–продукт, а также необходимостью учитывать аллергенность ингредиентов. Всё это усложняет классические подходы к построению расписаний и требует применения интеллектуальных методов оптимизации.

Решение задачи невозможно без учёта всех особенностей реального производства. При этом важно не только найти допустимое расписание, удовлетворяющее всем жёстким ограничениям, но и стремиться к оптимальному — с точки зрения минимизации времени простоя оборудования, сокращения количества санитарных моек, разумного распределения нагрузки и оперативного выполнения приоритетных заказов. 

Математическая модель задачи охватывает все ключевые параметры: объём задания, тип продукта, скорость линии, продолжительность и порядок выполнения, потребность в переналадке. В качестве целевой функции выступает минимизация совокупных издержек (времени мойки, длительности расписания, разрывов между заданиями и др.), что соответствует бизнес-целям предприятия.

Сложность задачи и многообразие параметров делают её хорошим кандидатом для применения современных методов искусственного интеллекта — в частности, решателей на основе ограничений. В следующей главе будет представлена реализация данной модели с использованием системы Timefold Solver — одного из ведущих инструментов в области гибридной оптимизации. Будут рассмотрены основные принципы его работы, структура планировщика, подход к формализации ограничений и методы поиска оптимального решения. Это позволит перейти от теоретической постановки задачи к её практическому решению в условиях реального производства.
