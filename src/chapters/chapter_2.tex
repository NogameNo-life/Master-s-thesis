\chapter{ ГЛАВА 2. Описание задачи планирования}
\label{ch:chapter2}

\section{Постановка задачи и описание модели}
Целью данной работы является разработка алгоритма, обеспечивающего оптимальное распределение заданий по фасовочным линиям на производстве глазированных сырков. Решение должно учитывать производственные ограничения, минимизировать общее время выполнения заказов, количество моек оборудования и простои между партиями.

\section{Что нужно оптимизировать}

\begin{enumerate}
	\item Минимизировать суммарную длительность всех переналадок (минимизация затрат).
	\item Минимизировать простои оборудования.
	\item Партии с приоритетом должны быть произведены до истечения даты выработки.
	\item Учет временных ограничений простоев оборудования и персонала (поломки, обеденный перерыв).
	\item Учет временных ограничений моек линии.
\end{enumerate}

На входе планировщика – разбивка производственного заказа по партиям. На выходе – разбивка партий по линиям.

\section{Исходные данные о работе производства}

\subsection{Производственные ресурсы}
На предприятии установлены шесть фасовочных линий.  Линии 1,2,3(GSL6) производительными чем линии 4,5,6(GSL8). Скорость работы зависит как от линии, так и от типа продукта.

\subsection{Типы продукции}

По техпроцессу производства глазированных сырков можно выделить три типа :
\begin{table}[h]
\centering
\caption{Типы продукции}
\begin{tabularx}{\textwidth}{|l|X|}
\hline
\textbf{Тип сырка} & \textbf{Комментарий} \\
\hline
Классика                   & Основной тип, включает множество начинок и глазурей (флоупак, 40 г). \\
\hline
Плюш                       & Производится только на первой линии (фольга, 45 г). \\
\hline
Стержень (Топ, Творобушки) & Требует переналадок, связанных с мойкой фруктопитателя (флоупак, 40 г). \\
\hline
Кактус                     & Требует переналадок до и после, длительностью около 3--4 часов (флоупак, 40 г). \\
\hline
\end{tabularx}
\end{table}

\begin{table}[h]
\centering
\caption{Производительность линий}
\begin{tabularx}{\textwidth}{|l|c|X|}
\hline
\textbf{№ Линии} & \textbf{ Кол/мин} & \textbf{Вид сырка} \\
\hline
1 (GSL6)     & 200—208 & Классика, кактус, плюш \\
\hline
2 (GSL6)     & 196     & Классика, кактус \\
\hline
3 (GSL6)     & 206     & Классика, кактус \\
\hline
4,5,6 (GSL8) & 220     & Классика \\
\hline
4,5,6 (GSL8) & 198     & Стержень \\
\hline
\end{tabularx}
\end{table}

Некоторые продукты содержат аллергены (арахис, фисташка, фундук, кокос-миндаль), и должны планироваться в конце смены, после чего выполняется обязательная мойка оборудования.

\subsection{Типы переналадок и моек}

\begin{table}[h]
\centering
\caption{Временные затраты на переналадки}
\begin{tabularx}{\textwidth}{|l|c|X|}
\hline
\textbf{Тип перехода} & \textbf{Время} & \textbf{Комментарий} \\
\hline
Стержень $\rightarrow$ Классика & 4–5 ч & Полная мойка и переналадка всех компонентов \\
Смена начинки в стержне & 50 мин & Внутри одного типа продукции \\
Смена глазури с мойкой (аллергены) & 1.5 ч & Обязательна после аллергенов \\
Смена глазури без мойки & 30–50 мин & При совместимых глазурях \\
Смена творожной массы без смены глазури & 20 мин & Лёгкая переналадка \\
Плюш $\rightarrow$ Классика & -- & Требует уточнения \\
\hline
\end{tabularx}
\end{table}

Также:
\begin{itemize}
  \item Обязательная мойка линий — раз в сутки:
  \begin{itemize}
    \item GSL6: 3–4 ч;
    \item GSL8: 4–5 ч.
  \end{itemize}
  \item Дополнительная мойка проводится до и после производства продукции с аллергенами, а также перед/после ``Кактуса'' (по 3 часа).
\end{itemize}

\subsection{Ограничения и особенности}
\begin{itemize}
  \item Максимальный объём производства — до 44 тонн в сутки.
  \item Очередность производств должна учитывать ограничения на смену начинок и глазурей.
  \item Варка глазури ограничена мощностью трёх мельниц (всего 6 тонн/сутки), каждая из которых имеет свою длительность варки и объём.
\end{itemize}

\subsection{Входные и выходные данные планировщика}
\begin{itemize}
  \item \textbf{На входе:} разбивка производственного заказа по партиям, каждая из которых характеризуется:
  \begin{itemize}
    \item типом продукта (Классика, Стержень, Плюш),
    \item глазурью,
    \item начинкой,
    \item объёмом (в кг),
    \item датой выработки,
    \item приоритетом выполнения.
  \end{itemize}
  \item \textbf{На выходе:} расписание партий по фасовочным линиям с учётом минимизации простоев и переналадок.
\end{itemize}

