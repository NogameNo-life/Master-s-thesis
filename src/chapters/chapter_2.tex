\chapter{ Описание задачи планирования}
\label{ch:chapter2}

\section{Постановка задачи и описание модели}
Целью данной работы является разработка алгоритма, обеспечивающего оптимальное распределение заданий по фасовочным линиям на производстве глазированных сырков. Решение должно учитывать производственные ограничения, минимизировать общее время выполнения заказов, количество моек оборудования и простои между партиями.

Что нужно оптимизировать:

\begin{enumerate}
	\item Минимизировать суммарную длительность всех переналадок (минимизация затрат).
	\item Минимизировать простои оборудования.
	\item Партии с приоритетом должны быть произведены до истечения даты выработки.
	\item Учет временных ограничений простоев оборудования и персонала (поломки, обеденный перерыв).
	\item Учет временных ограничений моек линии.
\end{enumerate}

На входе планировщика – разбивка производственного заказа по партиям. На выходе – разбивка партий по линиям.

\section{Исходные данные о работе производства}

На предприятии установлены шесть фасовочных линий.  Линии 1,2,3(GSL6) менее производительные чем линии 4,5,6(GSL8). Скорость работы зависит как от линии, так и от типа продукта. По техпроцессу производства глазированных сырков можно выделить несколько типов, которые приведены в таблице \ref{table:ProductsType}.
\begin{table}[h]
\centering
\caption{Типы продукции}
\begin{tabularx}{\textwidth}{|l|X|}
\hline
\textbf{Тип продукта} & \textbf{Комментарий} \\
\hline
Классика                   & Основной тип, включает множество начинок и глазурей (флоупак, 40 г). \\
\hline
Плюш                       & Производится только на первой линии (фольга, 45 г). \\
\hline
Стержень (Топ, Творобушки) & Требует переналадок, связанных с мойкой фруктопитателя (флоупак, 40 г). \\
\hline
Кактус                     & Требует переналадок до и после, длительностью около 3--4 часов (флоупак, 40 г). \\
\hline
\end{tabularx}
\label{table:ProductsType}
\end{table}

\section{Ограничения и особенности}
\begin{itemize}
  \item Максимальный объём производства — до 44 тонн в сутки.
  \item Очередность производств должна учитывать ограничения на смену начинок и глазурей.
  \item Варка глазури ограничена мощностью трёх мельниц (всего 6 тонн/сутки), каждая из которых имеет свою длительность варки и объём.
  \item Линия обязательно должна полностью мыться раз в сутки.
  \item Продолжительность –- 3-4 часа для линий 1,2,3 и 4-5 часов для линий с фруктопитателем 4,5,6.
  \item Необходимо выполнить мойку после аллергенов (сырков с арахисом, фисташками, фундуком, кокосом-миндалем, картошкой(в составе яйцо)и бискотти).
  \item Дополнительная мойка проводится до и после производства продукции с аллергенами, а также перед/после ``Кактуса'' (по 3 часа).
\end{itemize}

Основные типы переналадок и моек:

\begin{table}[h]
\centering
\caption{Типы переналадок}
\begin{tabularx}{\textwidth}{|l|X|}
\hline
\textbf{Тип переналадки} & \textbf{Мойка} \\
\hline
Стержень-классика                                                    & 2.5 часа. \\
\hline
Стержень смена наполнителя                                           & 50 минут. \\
\hline
Смена глазури с мойкой (аллергены)                                   & 1.5 часа. \\
\hline
Смена глазури без мойки                                              & 40 минут. \\
\hline
Без смены глазури                                                    & 20 минут. \\
\hline
Плюш-классика                                                        &  2,5-3 часа.\\
\hline
Стержень с начинкой на стержень без начинки                          & 40 минут. \\
\hline
Смена пленки                                                         & 5-10 минут. \\
\hline
\end{tabularx}
\label{table:CleaningTimeBetweenProducts}
\end{table}

\section{Формализация задачи}

Для формализации задачи используется объектно-ориентированная модель, в которую входят следующие ключевые сущности: Продукт (Product), задача (Job),  фасовочная линия (Line).

Продукт (Product) — характеризуется типом сырка (ProductType) и видом глазури (Glaze). Примеры типов: CLASSIC, ROD, PLUSH, CACTUS. От типа продукта зависят технологические ограничения и допустимые линии фасовки.

Задача (Job) — упаковка определённого количества сырков одного продукта.

Фасовочная линия (Line) — производственная линия, на которую могут быть назначены задания. Каждая линия имеет уникальное имя и список заданий, упорядоченных по времени исполнения.

В процессе упаковки должны соблюдаться как жёсткие (hard), так и мягкие (soft) ограничения:

Жёсткие ограничения предпологают допустимость линии по типу продукта:
\begin{itemize}
        \item ROD можно фасовать только на линиях 4, 5, 6.
        \item PLUSH фасуется только на линии 1.
        \item CACTUS фасуется только на линиях 1, 2, 3.
        \item CLASSIC допускается линиях 1, 2, 3, 6.
\end{itemize}

Также необходимо учитывать очередность заданий на линии то есть задания на линии должны следовать друг за другом без пересечений и ограничения на время мойки.

Мягкие ограничения включают в себя:

\begin{enumerate}
    \item Минимизация количества моек: предпочтительно упаковывать подряд задания с одинаковым продуктом или глазурью.
    \item Группировка продукции по ID: желательно, чтобы все задания одного заказа шли последовательно, без разрывов и лишних моек.
    \item Равномерная загрузка линий: линии должны быть загружены равномерно, без перегрузки или простоя.
\end{enumerate}

\subsection{Математическая формализация задачи}

Длительность выполнения задания (duration) рассчитывается как отношение количества сырков к скорости работы линии для конкретного типа продукта:

\begin{equation}
    duration = \frac{quantity}{lineSpeed}
\end{equation}

Скорости линий задаются индивидуально в зависимости от типа продукта:

\begin{table}[h]
\centering
\caption{Производительность линий}
\begin{tabularx}{\textwidth}{|l|c|X|}
\hline
\textbf{№ Линии} & \textbf{ Кол/мин} & \textbf{Тип продукта} \\
\hline
1,2,3 (GSL6) & 192—198 & Классика \\
\hline
6 (GSL8)     & 240     & Классика \\
\hline
1 (GSL6)     & 164     & Плюш \\
\hline
1,2,3 (GSL6) & 186-188 & Кактус \\
\hline
4,5,6 (GSL8) & 198     & Стержень \\
\hline
\end{tabularx}
\end{table}

Пусть:
\begin{itemize}
    \item $J =\{j_1, j_2...j_n\}$ -- множество задач,
    \item $ L = \{l_1, _2...l_m\}$ -- множество линий,
    \item $type(j_i)$ -- тип продукта,
    \item $lineSpeed(l_k,type(j_i))$ -- скорость фасовки.
    \item $q_i$ -- объём задания $j_i$ (в штуках).
    \item $s_{k,t}$ -- скорость линии $l_k$ для типа продукта $t$ в шт/мин.
    \item $d_{i,k} = frac{q_i}{s_k, type(j_i)}$ -- длительность выполнения $j_i$ на линии $l_k$, если линия допустима.
    \item $c_{i,j} \in \{0, C\}$ -- время чистки между двумя заданиями, зависящее от различий в продукте или глазури (например, $C = 20$ минут).
    \item $x_{i,k} \in \{0, 1\}$ -- бинарная переменная, равная 1, если задание $j_i$ назначено на линию $l_k$.
    \item $start_i$ -- время начала выполнения задания $j_i$.
    \item $end_i = start_i+ d_{i,k}$ -- время окончания задания.
\end{itemize}

\begin{equation}
    duration(j_i, l_k) = \frac{quantity(j_i)}{lineSpeed(l_k,type(j_i))}
\end{equation}

\begin{equation}
    cleaningTime(j_i, j_i + 1)
    \label{eq:cleaning_time}
\end{equation}

\noindent Необходимо найти отображение:

\begin{equation}
    A : J \rightarrow L \times N
    \label{eq:find}
\end{equation}

\noindent где, $A(j_i) = (l_k,p)$ означает, что задание $j_i$ выполняется на линии $l_k$  в позиции $p$, при этом минимизируется:

\begin{equation}
    Objective = w_1 \dot totalCleaningTime + w_2 \dot makespan + w_3 + splitOrders
\end{equation}

\noindent где $w_1, w_2, w_3$ -- веса для приоритетов. Целевая функция \eqref{eq:target_function}

\begin{equation}
    min \left(w_1 \cdot \sum_{i,j}c_{i,j} + w_2 \cdot \sum_{orders} splitPenalty \right)
    \label{eq:target_function}
\end{equation}

\noindent где:

\begin{itemize}
    \item $\sum_{c_{i,j}}$ -- сумма затрат времени на мойку;
    \item $max end_i$ -- $makespan$ (длительность общего расписания);
    \item $splitPenalty$ -- штраф за разбиение заданий одного заказа.
\end{itemize}

Таким образом, задача сводится к оптимальному распределению множества задач $J$ по доступным линиям $L$ с учётом технологических ограничений, временных зависимостей и множества целей оптимизации. Формально, требуется найти такую функцию назначения $A$, которая минимизирует целевую функцию \eqref{eq:target_function}, обеспечивая допустимость решений по всем жёстким ограничениям.

Данная задача относится к классу задач дискретной оптимизации. Для её решения используется подход на основе ограничений (constraint satisfaction), реализованный в системе Timefold Solver. Этот подход позволяет формализовать как жёсткие, так и мягкие ограничения в виде ограничений и штрафов, а затем эффективно искать приближённое оптимальное решение с помощью метаэвристик, таких как Late Acceptance, Simulated Annealing и других.

В следующем разделе будет описана реализация модели в Timefold, включая определение планируемых сущностей, ограничений и конфигурации решателя.

