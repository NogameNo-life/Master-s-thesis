\chapter{ Описание задачи планирования}
\label{ch:chapter2}

\section{Постановка задачи и описание модели}
Целью данной работы является разработка алгоритма, обеспечивающего оптимальное распределение заданий по фасовочным линиям на производстве глазированных сырков. Решение должно учитывать производственные ограничения, минимизировать общее время выполнения заказов, количество моек оборудования и простои между партиями.

\section{Что нужно оптимизировать}

\begin{enumerate}
	\item Минимизировать суммарную длительность всех переналадок (минимизация затрат).
	\item Минимизировать простои оборудования.
	\item Партии с приоритетом должны быть произведены до истечения даты выработки.
	\item Учет временных ограничений простоев оборудования и персонала (поломки, обеденный перерыв).
	\item Учет временных ограничений моек линии.
\end{enumerate}

На входе планировщика – разбивка производственного заказа по партиям. На выходе – разбивка партий по линиям.

\section{Исходные данные о работе производства}

\subsection{Производственные ресурсы}
На предприятии установлены шесть фасовочных линий.  Линии 1,2,3(GSL6) производительными чем линии 4,5,6(GSL8). Скорость работы зависит как от линии, так и от типа продукта.

\subsection{Типы продукции}

По техпроцессу производства глазированных сырков можно выделить три типа :
\begin{table}[h]
\centering
\caption{Типы продукции}
\begin{tabularx}{\textwidth}{|l|X|}
\hline
\textbf{Тип сырка} & \textbf{Комментарий} \\
\hline
Классика                   & Основной тип, включает множество начинок и глазурей (флоупак, 40 г). \\
\hline
Плюш                       & Производится только на первой линии (фольга, 45 г). \\
\hline
Стержень (Топ, Творобушки) & Требует переналадок, связанных с мойкой фруктопитателя (флоупак, 40 г). \\
\hline
Кактус                     & Требует переналадок до и после, длительностью около 3--4 часов (флоупак, 40 г). \\
\hline
\end{tabularx}
\end{table}

Некоторые продукты содержат аллергены (арахис, фисташка, фундук, кокос-миндаль), и должны планироваться в конце смены, после чего выполняется обязательная мойка оборудования.

\subsection{Типы переналадок и моек}

\begin{enumerate}
  \item Стержень-классика – 2.5 часа. 
  \item Стержень смена наполнителя – 50 минут. Смена начинки в стержнях занимает 50 минут. Обычно фасуется в порядке (манго-клубника-малина-шоколад-карамель-арахис-фундук).
  \item Смена глазури с мойкой (аллергены) – 1.5 часа (кокос-миндаль – фисташка). Аллергены всегда в конце партии.
  \item Смена глазури без мойки – 30-50 минут (коровка – аленка –ваниль).
  \item Без смены глазури – 20 минут. Если глазурь одинаковая и нужно заменить только творожную массу, то это занимает 20 минут (пример: переход ванили на кофе-карамель).
  \item Плюш-классика – необходимо уточнить.
\end{enumerate}

\subsection{Мойка линий}

\begin{enumerate}
  \item Линия обязательно должна полностью мыться раз в сутки.
  \item Продолжительность –- 3-4 часа для линий 1,2,3 и 4-5 часов для линий с фруктопитателем 4,5,6.
  \item Необходимо выполнить мойку после аллергенов (сырков с арахисом, фисташками, фундуком, кокосом-миндалем).
\end{enumerate}

Также:
\begin{itemize}
  \item Обязательная мойка линий — раз в сутки:
  \item Дополнительная мойка проводится до и после производства продукции с аллергенами, а также перед/после ``Кактуса'' (по 3 часа).
\end{itemize}

\subsection{Ограничения и особенности}
\begin{itemize}
  \item Максимальный объём производства — до 44 тонн в сутки.
  \item Очередность производств должна учитывать ограничения на смену начинок и глазурей.
  \item Варка глазури ограничена мощностью трёх мельниц (всего 6 тонн/сутки), каждая из которых имеет свою длительность варки и объём.
\end{itemize}

\subsection{Входные и выходные данные планировщика}
\begin{itemize}
  \item \textbf{На входе:} разбивка производственного заказа по партиям, каждая из которых характеризуется:
  \begin{itemize}
    \item типом продукта (Классика, Стержень, Плюш),
    \item глазурью,
    \item начинкой,
    \item объёмом (в кг),
    \item датой выработки,
    \item приоритетом выполнения.
  \end{itemize}
  \item \textbf{На выходе:} расписание партий по фасовочным линиям с учётом минимизации простоев и переналадок.
\end{itemize}

\subsection{Формализация задачи}

Для формализации задачи используется объектно-ориентированная модель, в которую входят следующие ключевые сущности:

\begin{itemize}
    \item Продукт (Product) — характеризуется типом сырка (ProductType) и видом глазури (Glaze). Примеры типов: CLASSIC, ROD, PLUSH, CACTUS. От типа продукта зависят технологические ограничения и допустимые линии фасовки.
    \item Задание (Job) — упаковка определённого количества сырков одного продукта. Каждое задание имеет:
    \begin{itemize}
        \item уникальный идентификатор,
        \item ссылку на Product,
        \item объём фасовки (quantity),
        \item вычисляемую длительность выполнения (duration),
        \item ссылку на линию (Line), которую назначает планировщик.
    \end{itemize}
    \item Фасовочная линия (Line) — производственная линия, на которую могут быть назначены задания. Каждая линия имеет уникальное имя и список заданий, упорядоченных по времени исполнения.
\end{itemize}

\section{Ограничения задачи}

В процессе упаковки должны соблюдаться как жёсткие (hard), так и мягкие (soft) ограничения:

\subsection{ Жёсткие ограничения}

\begin{itemize}
    \item Допустимость линии по типу продукта
    \begin{itemize}
        \item ROD можно фасовать только на линиях 4, 5, 6.
        \item PLUSH фасуется только на линии 1.
        \item CACTUS фасуется только на линиях 1, 2, 3.
        \item CLASSIC допускается на всех линиях.
    \end{itemize}
    \item Очередность заданий на линии: задания на линии должны следовать друг за другом без пересечений.
    \item Ограничения на время мойки: при смене продукта или глазури между двумя заданиями требуется вставка времени на мойку линии.
\end{itemize}

\subsection{Мягкие ограничения}

\begin{itemize}
    \item Минимизация количества моек: предпочтительно упаковывать подряд задания с одинаковым продуктом или глазурью.
    \item Группировка продукции по ID: желательно, чтобы все задания одного заказа шли последовательно, без разрывов и лишних моек.
    \item Равномерная загрузка линий: линии должны быть загружены равномерно, без перегрузки или простоя.
\end{itemize}

\subsection{Вычисление длительности задания}

Длительность выполнения задания (duration) рассчитывается как отношение количества сырков к скорости работы линии для конкретного типа продукта:

\begin{equation}
    duration = \frac{quantity}{lineSpeed}
\end{equation}

Скорости линий задаются индивидуально в зависимости от типа продукта:


\begin{table}[h]
\centering
\caption{Производительность линий}
\begin{tabularx}{\textwidth}{|l|c|X|}
\hline
\textbf{№ Линии} & \textbf{ Кол/мин} & \textbf{Вид сырка} \\
\hline
1 (GSL6)     & 200—208 & Классика, кактус, плюш \\
\hline
2 (GSL6)     & 196     & Классика, кактус \\
\hline
3 (GSL6)     & 206     & Классика, кактус \\
\hline
4,5,6 (GSL8) & 220     & Классика \\
\hline
4,5,6 (GSL8) & 198     & Стержень \\
\hline
\end{tabularx}
\end{table}

\subsection{Математическая формализация задачи}

Пусть:
\begin{itemize}
    \item $J =\{j_1, j_2...j_n\}$ -- множество заданий,
    \item $ L = \{l_1, _2...l_m\}$ -- множество линий,
    \item $type(j_i)$ -- тип продукта,
    \item $lineSpeed(l_k,type(j_i))$ -- скорость фасовки.
    \item $q_i$ -- объём задания $j_i$ (в штуках).
    \item $s_{k,t}$ -- скорость линии $l_k$ для типа продукта $t$ в шт/мин.
    \item $d_{i,k} = frac{q_i}{s_k, type(j_i)}$ -- длительность выполнения $j_i$ на линии $l_k$, если линия допустима.
    \item $c_{i,j} \in \{0, C\}$ -- время чистки между двумя заданиями, зависящее от различий в продукте или глазури (например, $C = 20$ минут).
    \item $x_{i,k} \in \{0, 1\}$ -- бинарная переменная, равная 1, если задание $j_i$ назначено на линию $l_k$.
    \item $start_i$ -- время начала выполнения задания $j_i$.
    \item $end_i = start_i+ d_{i,k}$ -- время окончания задания.
\end{itemize}

\begin{equation}
    duration(j_i, l_k) = \frac{quantity(j_i)}{lineSpeed(l_k,type(j_i))}
\end{equation}

\begin{equation}
    cleaningTime(j_i, j_i + 1)
    \label{eq:cleaning_time}
\end{equation}

\noindent Необходимо найти отображение:

\begin{equation}
    A : J \rightarrow L \times N
    \label{eq:find}
\end{equation}

\noindent где, $A(j_i) = (l_k,p)$ означает, что задание $j_i$ выполняется на линии $l_k$  в позиции $p$, при этом минимизируется:

\begin{equation}
    Objective = w_1 \dot totalCleaningTime + w_2 \dot makespan + w_3 + splitOrders
\end{equation}

\noindent где $w_1, w_2, w_3$ -- веса для приоритетов. Целевая функция \eqref{eq:target_function}

\begin{equation}
    min \left(w_1 \cdot \sum_{i,j}c_{i,j} + w_2 \cdot \sum_{orders} splitPenalty \right)
    \label{eq:target_function}
\end{equation}

\noindent где:

\begin{itemize}
    \item $\sum_{c_{i,j}}$ -- сумма затрат времени на мойку;
    \item $max end_i$ -- $makespan$ (длительность общего расписания);
    \item $splitPenalty$ -- штраф за разбиение заданий одного заказа.
\end{itemize}

