\chapter*{Введение}
\addcontentsline{toc}{chapter}{Введение}

В современных условиях стремительного технологического прогресса, цифровизации и усиления конкурентной борьбы на глобальном рынке, оптимизация производственных процессов становится одной из ключевых задач для предприятий, ориентированных на эффективность, гибкость и устойчивое развитие. Современное производство требует не только высокого качества выпускаемой продукции, но и способности быстро адаптироваться к изменениям спроса, колебаниям на рынке сырья, ужесточающимся требованиям к безопасности и устойчивости. В этой связи особую актуальность приобретает задача повышения эффективности ключевых звеньев производственной цепочки, к которым, безусловно, относятся фасовочные линии.

Фасовочные линии выполняют критически важную функцию на завершающих этапах производственного цикла: они обеспечивают дозировку, упаковку и подготовку продукции к транспортировке и реализации. От их корректной и слаженной работы зависят такие параметры, как точность фасовки, минимизация потерь, снижение простоев, а также соблюдение санитарных, технологических и логистических норм. Неправильное планирование загрузки линий, несвоевременное переключение между продуктами или несогласованность в логистике могут привести к значительным издержкам, снижению производительности и даже к ухудшению качества продукции.

Одной из ключевых задач в данной области является планирование загрузки фасовочных линий, с учетом множества реальных ограничений: типа продукции, необходимости мойки оборудования при смене аллергенов, ограниченного числа доступных производственных линий, временных окон, а также минимизации времени переналадки. Эти ограничения накладываются на необходимость поддержания высокой загрузки оборудования и соблюдения сроков выполнения заказов. Задача планирования в такой постановке относится к классу задач дискретной оптимизации и отличается высокой сложностью при увеличении количества производственных единиц, типов продукции и технологических условий.

Решение подобных задач традиционно требует значительных усилий со стороны специалистов, обладающих знаниями в области операционного планирования, математики, а также программирования. При этом даже наличие формализованной модели не гарантирует быстрой генерации качественных решений, особенно если используются универсальные инструменты без специфической адаптации под предметную область. В этой связи возникает потребность в применении специализированных интеллектуальных систем, способных автоматически или полуавтоматически формировать расписания, учитывающие все технологические и логистические ограничения.

Одним из современных программных решений, заслуживающих особого внимания в данной области, является \textbf{Timefold Solver} — высокоэффективный инструмент планирования и оптимизации на основе декларативного подхода. Он реализует методологию ограничения на основе Constraint Streams, что позволяет описывать бизнес-ограничения в виде потоков ограничений, легко читаемых и расширяемых. Timefold сочетает мощные эвристики и метаэвристики (такие как локальный поиск, симулированный отжиг, табу-поиск и др.) с гибкой архитектурой, позволяющей масштабировать решения от прототипов до промышленных систем.

Вместе с тем, несмотря на широкие возможности, создание эффективных моделей и описание ограничений в Timefold требует достаточно глубокого понимания как предметной области, так и архитектуры самого фреймворка. Написание сложных ограничений на Java — процесс, сопряжённый с трудоёмкой отладкой, необходимостью точного следования API и значительным временем на тестирование. В условиях постоянного изменения требований бизнеса и необходимости быстрой итерации над моделью возникает необходимость в новых инструментах, упрощающих и ускоряющих этот процесс.

С развитием технологий искусственного интеллекта, особенно в области генеративных нейросетей, появилась возможность автоматизировать часть процессов, ранее выполнявшихся вручную. В частности, комбинирование технологий машинного перевода и генерации программного кода на основе естественного языка открывает новые горизонты для ускоренной разработки моделей ограничений. Это особенно актуально в промышленной среде, где инженеры и технологи могут формулировать задачи на родном языке, а система автоматически преобразует их в исполнимый код, адаптированный под конкретный планировщик.

Таким образом, настоящая магистерская диссертация посвящена исследованию и практической реализации подхода к оптимизации фасовочных линий с использованием Timefold Solver, а также расширению возможностей этого инструмента за счёт внедрения нейросетевых моделей для генерации ограничений на основе естественного языка. В рамках работы рассматривается формализация задачи планирования, реализация программной архитектуры системы автоматизации генерации ограничений, а также проводится анализ преимуществ и ограничений предложенного подхода. Отдельное внимание уделяется вопросам адаптации решения под реальные производственные условия, обеспечению масштабируемости и безопасности при локальном развёртывании.

Ожидается, что результаты данного исследования не только подтвердят практическую применимость Timefold Solver для задач производственного планирования, но и продемонстрируют потенциал интеграции инструментов искусственного интеллекта в повседневную деятельность специалистов по планированию. Предлагаемый подход может существенно ускорить внедрение систем интеллектуального планирования в промышленность, снизить требования к технической квалификации пользователей и повысить адаптивность предприятий к меняющимся условиям рынка.
