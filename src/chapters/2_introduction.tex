\chapter*{Введение}
\addcontentsline{toc}{chapter}{Введение}

В условиях стремительного развития технологий и глобализации рынка, оптимизация производственных процессов становится одной из ключевых задач для предприятий, стремящихся сохранить конкурентоспособность и увеличить свою долю на рынке. Особое внимание в этом контексте заслуживает оптимизация фасовочных линий, которые играют критическую роль в производственных цепочках множества отраслей — от пищевой и фармацевтической до косметической и химической. Эффективная работа фасовочных линий не только влияет на объемы производства, но и определяет качество конечного продукта, что в свою очередь имеет прямое отношение к удовлетворенности потребителей и репутации компании.

Оптимизация фасовочных линий включает в себя множество аспектов: от сокращения времени простоя оборудования до повышения точности дозирования и улучшения логистики. Современные предприятия сталкиваются с необходимостью адаптации к быстро меняющимся условиям рынка, что требует гибкости в производственных процессах. Эффективное управление ресурсами и планирование позволяет значительно снизить затраты, повысить производительность и улучшить качество продукции. Таким образом, задача оптимизации становится не просто желательной, а жизненно необходимой для достижения устойчивого роста и развития бизнеса.

Среди современных инструментов, используемых для решения задач оптимизации, особое место занимают программные решения, такие как Timefold Solver. Этот инструмент представляет собой мощный планировщик, который позволяет моделировать и оптимизировать производственные процессы с учетом множества факторов, включая ограничения по ресурсам, временные рамки и требования к качеству. Timefold Solver использует алгоритмы оптимизации, способные находить эффективные решения даже в сложных ситуациях с большим количеством переменных. Его применение позволяет значительно сократить время на планирование и повысить точность прогнозов, что в свою очередь способствует более рациональному использованию ресурсов и снижению операционных затрат.

Таким образом, данная магистерская диссертация будет посвящена исследованию методов и инструментов оптимизации фасовочных линий, с акцентом на использование Timefold Solver как примера современного подхода к решению задач планирования. В рамках работы будут рассмотрены основные принципы работы фасовочных линий, выявлены ключевые проблемы, с которыми сталкиваются предприятия, а также предложены рекомендации по их решению с использованием современных технологий. Ожидается, что результаты данного исследования смогут внести значительный вклад в развитие практических аспектов управления производственными процессами и оптимизации ресурсов на предприятиях различных отраслей.