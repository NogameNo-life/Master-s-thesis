\chapter{Применение нейросетевых моделей для генерации ограничений в планировщике и разработка пользовательского интерфейса}
\label{ch:chapter4}

Современные методы оптимизации производства всё чаще дополняются подходами искусственного интеллекта, включая использование нейронных сетей. В рамках данной работы была реализована система, в которой генерация ограничений для планировщика осуществляется с помощью языковых моделей (LLM), обрабатывающих текстовые описания ограничений (промпты). Для удобства взаимодействия с системой был разработан специализированный веб-интерфейс на базе React и Next.js.

\section{Обзор нейросетевых моделей для генерации ограничений}

Задача генерации ограничений в классическом планировании (например, в Timefold Solver) ограничения описываются вручную на Java или Kotlin с использованием API ограничения (constraint streams). Это требует от разработчика глубоких знаний в предметной области и программировании. Использование языковых моделей позволяет частично автоматизировать процесс генерации таких ограничений на основании текстового описания.

Для решения задачи генерации ограничений были рассмотрены следующие типы языковых моделей:

\begin{itemize}
    \item \textbf{Переводчики (NMT)}: используются для перевода исходного промпта с русского на английский язык (например, модели MarianMT или T5 от Hugging Face).
    \item \textbf{Кодогенераторы (Code LLMs)}: специализированные модели, такие как CodeLlama, DeepSeekCoder, StarCoder или GPT-4, способные генерировать Java-код по текстовому описанию.
    \item \textbf{Модели общего назначения (ChatGPT, Claude, Gemini)}: используются как универсальные генераторы кода и логики ограничений.
\end{itemize}

Выбор конкретной модели зависит от таких факторов как качество перевода,влияет на точность интерпретации начального промпта, контекстное понимание предметной области. Также необходимо учитывать возможности развертывания, будет ли модель использоваться локально (например, в корпоративной среде) или через API.

\section{Архитектура разработанной системы}

Система состоит из следующих компонентов:

\begin{enumerate}
    \item \textbf{Пользовательский интерфейс (React/Next.js)}: предоставляет пользователю возможность ввести промпт на русском языке.
    \item \textbf{Сервис перевода}: выполняет перевод на английский язык (с использованием Hugging Face Transformers).
    \item \textbf{Сервис генерации кода}: обрабатывает английский промпт и возвращает сгенерированный Java-код с ограничением.
    \item \textbf{История взаимодействий}: сохраняет промпты, переводы и результат генерации для повторного использования и анализа.
\end{enumerate}

Интерфейс реализован с использованием Next.js и Tailwind CSS, включает следующие ключевые элементы:

\begin{itemize}
    \item Поле ввода на русском языке
    \item Кнопка перевода
    \item Окно просмотра переведённого промпта
    \item Кнопка генерации
    \item Окно вывода Java-кода
    \item История генераций
\end{itemize}

Перевод выполняется через Hugging Face API (модель Helsinki-NLP/opus-mt-ru-en). Генерация кода — через локальный сервер с установленной моделью CodeLlama или через облачный API (например, OpenAI или Replicate). Генерация кода — через локальный сервер с установленной моделью CodeLlama или через облачный API (например, OpenAI или Replicate).

\section{ Преимущества и ограничения подхода}

Преимущества:

\begin{itemize}
    \item Ускорение написания ограничений.
    \item Удобство для пользователей без глубоких знаний Java.
    \item Возможность экспериментировать с различными формулировками ограничений.
\end{itemize}

Ограничения:

\begin{itemize}
    \item Необходима валидация сгенерированного кода.
    \item Возможны ошибки перевода или некорректная логика.
    \item Модели могут не учитывать бизнес-контекст, если промпт недостаточно точен.
\end{itemize}