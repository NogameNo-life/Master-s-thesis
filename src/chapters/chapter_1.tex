\chapter{ ГЛАВА 1. СОВРЕМЕННЫЕ СИСТЕМЫ ПЛАНИРОВАНИЯ}
\label{ch:chapter1}

\section{ Обзор существующих систем}
В условиях современного производства и высоких требований к эффективности, выбор подходящей системы планирования становится критически важным для успешной оптимизации фасовочных линий. Существует множество программных решений, каждое из которых предлагает свои уникальные функции и алгоритмы для решения задач планирования. В данной главе будет представлен обзор наиболее популярных систем планирования, включая их ключевые особенности, преимущества и недостатки, а также будет проведено сравнение с Timefold Solver.

 \begin{enumerate}
    \item \noindent \textbf{Google OR-Tools}
OR-Tools (Operations Research Tools) — это библиотека с открытым исходным кодом от Google, предназначенная для решения задач оптимизации, включая линейное программирование, поиск с ограничениями, маршрутизацию и планирование.

\vspace{6mm}

\noindent \textbf{Преимущества:}
\begin{itemize}
    \item Поддержка множества языков программирования (C++, Python, Java, C\#).
    \item Хорошо документирована и активно поддерживается.
    \item Широкий спектр решаемых задач: от маршрутизации до составления расписаний.
Интеграция с другими инструментами Google.
\end{itemize} 

\vspace{6mm}

\item \noindent \textbf{IBM ILOG CPLEX}

CPLEX — это коммерческий инструмент для решения задач линейного, целочисленного и квадратичного программирования. Он также поддерживает поиск с ограничениями.
\vspace{6mm}

\noindent \textbf{Преимущества:}
\begin{itemize}
    \item Высокая производительность и точность.
    \item Поддержка сложных бизнес-задач.
    \item Интеграция с другими продуктами IBM.
Интеграция с другими инструментами Google.
\end{itemize}

\vspace{6mm}

\noindent \textbf{Недостатки:}
\begin{itemize}
\item Дорогостоящий (не подходит для небольших проектов).
\item Закрытый исходный код.
\item Сложность в освоении для новичков.
\end{itemize}

\vspace{6mm}

\item \noindent \textbf {Gurobi}

Gurobi — это коммерческий решатель для задач линейного, целочисленного и квадратичного программирования. Он также поддерживает поиск с ограничениями.
\vspace{6mm}

\noindent \textbf{Преимущества:}
\begin{itemize}
    \item Высокая скорость и точность.
    \item Поддержка множества языков программирования.
    \item Хорошо подходит для крупных предприятий.
Интеграция с другими инструментами Google.
\end{itemize}

\vspace{6mm}

\noindent \textbf{Недостатки:}
\begin{itemize}
\item Высокая стоимость.
\item Закрытый исходный код.
\item Ориентирован на линейное программирование, что может быть избыточным для некоторых задач.
\end{itemize}

\vspace{6mm}

\item \noindent \textbf{MiniZinc}

MiniZinc — это язык моделирования и среда для решения задач поиска с ограничениями. Он поддерживает множество решателей (включая Gecode, Chuffed и другие).
\vspace{6mm}

\noindent \textbf{Преимущества:}
\begin{itemize}
    \item Гибкость в выборе решателя.
    \item Подходит для академических и исследовательских задач.
    \item Открытый исходный код.
\end{itemize}

\vspace{6mm}

\noindent \textbf{Недостатки:}
\begin{itemize}
\item Требует знания языка моделирования MiniZinc.
\item Меньше готовых решений для бизнес-задач.
\end{itemize}

\vspace{6mm}

 \item \noindent \textbf {Choco Solver}

 Choco Solver — это библиотека для поиска с ограничениями на Java. Она ориентирована на задачи, связанные с ограничениями и оптимизацией.
\vspace{6mm}

\noindent \textbf{Преимущества:}
\begin{itemize}
    \item Легко интегрируется с Java-приложениями.
    \item Открытый исходный код.
    \item Подходит для задач средней сложности.
\end{itemize}

\vspace{6mm}

\noindent \textbf{Недостатки:}
\begin{itemize}
\item Меньше функциональности по сравнению с OptaPlanner.
\itemТребует глубокого понимания поиска с ограничениями.
\end{itemize}

\vspace{6mm}

\item \noindent \textbf {Z3 Solver}

  Z3 — это решатель SMT (Satisfiability Modulo Theories) от Microsoft, который используется для проверки моделей и решения задач с ограничениями.
\vspace{6mm}

\noindent \textbf{Преимущества:}
\begin{itemize}
    \item Мощный инструмент для формальной верификации и анализа.
    \item Поддержка логических ограничений.
    \item Открытый исходный код.
\end{itemize}

\vspace{6mm}

\noindent \textbf{Недостатки:}
\begin{itemize}
\item Сложность в освоении.
\item Не подходит для задач, требующих эвристик.
\end{itemize}

\vspace{6mm}

\item \noindent \textbf {LocalSolver}

  LocalSolver — это коммерческий инструмент для решения задач оптимизации, который использует эвристические методы.
  
\vspace{6mm}

\noindent \textbf{Преимущества:}
\begin{itemize}
    \item Высокая производительность.
    \item Поддержка нелинейных и сложных ограничений.
    \item Простота в использовании.
\end{itemize}

\vspace{6mm}

\noindent \textbf{Недостатки:}
\begin{itemize}
\item Высокая стоимость.
\item Закрытый исходный код.
\end{itemize}

\end{enumerate}

\section{ Сравнение с OptaPlanner и Timefold Solver:}

OptaPlanner и Timefold Solver выделяются своей гибкостью, поддержкой сложных эвристик и интеграцией с Java-экосистемой. Они подходят для широкого спектра задач, от составления расписаний до маршрутизации.
\vspace{6mm}

\noindent \textbf{Преимущества:}
\begin{itemize}
    \item Открытый исходный код (в случае OptaPlanner).
    \item Широкая поддержка сообщества.
    \item Готовые решения для бизнес-задач.
\end{itemize}

\vspace{6mm}

\noindent \textbf{Недостатки:}
\begin{itemize}
\item Требуют знаний Java.
\item Меньше поддержки для не-Java экосистем.
\end{itemize}

