\chapter*{ОБЩАЯ ХАРАКТЕРИСТИКА РАБОТЫ}
\addcontentsline{toc}{chapter}{ОБЩАЯ ХАРАКТЕРИСТИКА РАБОТЫ}
\label{ch:target}

Тема диссертации соответствует приоритетному направлению научно-технической деятельности согласно пункту 1 перечня приоритетных направлений научной, научно-технической и инновационной деятельности на 2021-2025 годы (Указ Президента Республики Беларусь от 07 мая 2020 г. № 156).

\medskip  % примерно 6pt
\bigskip  % примерно 12pt

\noindent \textbf{Цель работы:}

Улучшение эффективности производства за счет минимизации времени выполнения задач и оптимального использования ресурсов. Определить возможные улучшения и дальнейшие направления исследования. Оценить влияние оптимизации на производительность, время выполнения задач и использование ресурсов.
Также — исследование возможностей применения методов искусственного интеллекта для автоматизации разработки аналогичных планировщиков для других типов продукции. В частности, экспериментальная проверка подхода, при котором нейросети используются для генерации Java-кода ограничений планирования на основе текстового описания производственной задачи.
\noindent \textbf{Задачи:}
\begin{enumerate}
    \item Анализ существующих процессов:
Изучить текущие методы и процессы фасовки.
Определение ключевых ограничений и требований, таких как время, ресурсы, оборудование и персонал. Проведение сравнения с существующими методами планирования.
    \item  Моделирование задачи планирования:
Создать модель задачи планирования в Timefold Solver, учитывая специфику фасовки того или иного товара. Определить переменные, ограничения и целевую функцию для оптимизации.
    \item Разработка алгоритма оптимизации:
Использовать современные инструменты для решения задач оптимизации, для поиска наилучшего расписания.
Оптимизировать расписание с учетом минимизации времени выполнения задач, использования ресурсов и обеспечения высокой производительности.
    \item Реализация модели:
Реализовать разработанный алгоритм на Java с использованием Timefold Solver.
    \item Тестирование результатов:\
Протестировать эффективность разработанного алгоритма.
\item Исследование возможностей автоматизации разработки планировщиков с использованием ИИ:  
Протестировать две нейросетевые модели на сервере с GPU.  
Одна модель переводит текст описания задачи с русского на английский, вторая — генерирует Java-код ограничений (Constraint Streams) на основе переведенного описания.  
Оценить применимость такого подхода для автоматической генерации планировщиков для других типов продукции.
\item Внедрение и рекомендации:\
Предложить рекомендации по внедрению разработанного планирования в реальном производстве.
\end{enumerate}

\vspace{6mm}


\textit{Объектом исследования} являются фасовочные линии на производстве.

\vspace{6mm}

\textit{Предметом исследования} выступают методы и алгоритмы оптимизации в современных системах планирования, а также подходы к автоматической генерации кода с помощью искусственного интеллекта.

\vspace{6mm}

\noindent \textbf{Положения выносимые на защиту}
\begin{enumerate}
    \item Программная система составляюшая план фасовки производственного заказа глазированных сырков.
    \item Анализ результатов тестирования работы системы.
    \item Предварительное исследование возможностей генерации ограничений планирования с помощью нейросетей по текстовому описанию задачи.
\end{enumerate}

\vspace{6mm}

\noindent \textbf{Личный вклад соискателя}\\
\indent Разработка программной системы, получение результатов оценки планирования работы фасовочных линий.

\vspace{6mm}

\textit{Научная новизна} состоит в оценке эффективности использования современных систем планирования.

\vspace{6mm}

\noindent \textbf{Структура и объем диссертации}\\

Основное содержание работы изложено на 44 страницах машинописного текста, иллюстрировано 14 рисунками, 9 листингами кода, содержит 4 таблицы. Диссертация состоит из введения, общей характеристики работы, пяти глав, заключения, списка литературы. Общий объем работы – 57 страниц.
