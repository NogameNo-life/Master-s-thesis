% !TeX encoding = UTF-8 Unicode
% !TeX root = main.tex
% !TeX TS-program = pdflatex
%% (При смене движка необходимо удалить вспомогательные файлы *.aux *.brf *.log *.out *.synctex.gz *.toc)

\documentclass{thesisby}
\usepackage{etoolbox,ifxetex,ifluatex}
\usepackage[unicode,hypertexnames=false]{hyperref}

%%% Проверка используемого TeX-движка %%%
\ifboolexpr{bool{xetex} or bool{luatex}}{%
  \usepackage{fontspec}
  \PassOptionsToPackage{no-math}{fontspec}     % https://tex.stackexchange.com/a/26295/104425
  \usepackage{polyglossia}%[2014/05/21]        % Поддержка многоязычности

  % fonts and languages
  \defaultfontfeatures{Ligatures=TeX,Mapping=tex-text}

  \setmainlanguage[babelshorthands = true]{russian}
  \setotherlanguage{english}

  \setmainfont{Times New Roman}
  \setmonofont{Courier New}
  \setsansfont{Arial}

  \newfontfamily\cyrillicfont[Script=Cyrillic]{Times New Roman}
  \newfontfamily\cyrillicfontsf[Script=Cyrillic]{Arial}
  \newfontfamily\cyrillicfonttt[Script=Cyrillic]{Courier New}

  \newfontfamily\englishfont{Times New Roman}
  \newfontfamily\englishfontsf{Arial}
  \newfontfamily\englishfonttt{Courier New}

  \renewcommand{\UrlFont}{\small\rmfamily\tt}
}{%
  \usepackage[T1,T2A]{fontenc}
  \usepackage[utf8]{inputenc}
  \usepackage[english, russian]{babel}
  \usepackage{csquotes}
  \IfFileExists{pscyr.sty}{\usepackage{pscyr}}{}  % Подключение pscyr
}

% Для борьбы с переполнениями за счет разреженных слов в абзаце
\emergencystretch=25pt

\usepackage{enumitem}

\usepackage[
    language = auto,        % Получение языка из babel.
    autolang = other,       % Многоязычная библиография.
    defernumbers = true,    % Раздельная нумерация.
    style = gost-numeric,
    maxnames = 10,
    movenames = false,
    sorting = ynt
]{biblatex} % To load multiple bib files.

% Библиографический список в соответствии с ГОСТ.
\makeatletter
\ltx@iffilelater{biblatex-gost.def}{2017/02/01}%
{\toggletrue{bbx:gostbibliography}%
    \renewcommand*{\revsdnamepunct}{\addcomma}}{}
\makeatother

% Общий список.
\addbibresource{references.bib}

% Список публикаций соискателя.
\addbibresource{references.bib}
\DeclareSourcemap{
    \maps[datatype=bibtex, overwrite]{
        \map{
            \perdatasource{references.bib}
            \step[fieldset=KEYWORDS, fieldvalue=ys, append]
        }
    }
}

% Счётчики.
\usepackage[figure,table]{totalcount}   % Счётчик рисунков и таблиц.
\usepackage{totcount}                   % Пакет создания счётчиков на основе последнего номера подсчитываемого элемента (может требовать дважды компилировать документ).
\AtEveryBibitem{\stepcounter{citenum}\stepcounter{citenum_my}}

\usepackage{totpages}
\usepackage[abspage, user, lastpage]{zref}

\usepackage{microtype}

%for lists
\usepackage[ampersand]{easylist}
\ListProperties(Hide=100, Hang=false, Margin=0mm, Indent1=10.5mm, Indent2=15mm, Style*=-- ,
Style2*=$\bullet$ ,Style3*=$\circ$ ,Style4*=\tiny$\blacksquare$ )

\newenvironment{easylistNum}{
    \begin{easylist}
        \ListProperties(Hide1=0, Hang=false, Margin=0mm, Indent1=10.5mm, Indent2=15mm, Start1=1, Style*=, FinalMark={)})}
        {\ListProperties(Hide=100, Hang=false, Margin=0mm, Indent1=10.5mm, Indent2=15mm, Style*=-- , Style2*=$\bullet$ ,Style3*=$\circ$ ,Style4*=\tiny$\blacksquare$ )
    \end{easylist}}

\usepackage{amsmath, amssymb, amsfonts}
\usepackage{mathtools} % Use \rcases
\usepackage{longtable, array}
\usepackage{graphicx, epsfig}
\usepackage{tabularx}

\usepackage{algorithm}        % Для вставки псевдокода
\usepackage{algpseudocode}    % Для вставки псевдокода

% Русская традиция начертания греческих букв
\usepackage{upgreek} % Прямые греческие ради русской традиции (в формулах записывается \alpha как \upalpha и т.д.)

\usepackage{siunitx} % For Celsium sign only

\usepackage{tikz} % Language for producing vector graphics

\usepackage{listings}
\usepackage{xcolor}

\lstdefinelanguage{Java}{
  morekeywords={abstract, assert, boolean, break, byte, case, catch, char,
    class, const, continue, default, do, double, else, enum, extends,
    final, finally, float, for, goto, if, implements, import, instanceof,
    int, interface, long, native, new, package, private, protected, public,
    return, short, static, strictfp, super, switch, synchronized, this,
    throw, throws, transient, try, void, volatile, while},
  sensitive=true,
  morecomment=[l]{//},
  morecomment=[s]{/*}{*/},
  morestring=[b]",
}
\definecolor{codegreen}{rgb}{0,0.6,0}
\definecolor{codegray}{rgb}{0.5,0.5,0.5}
\definecolor{codepurple}{rgb}{0.58,0,0.82}
\definecolor{backcolour}{rgb}{0.95,0.95,0.92}
\lstset{
  language=Java,
  basicstyle=\ttfamily\footnotesize,
  keywordstyle=\color{blue},
  commentstyle=\color{codegreen},
  stringstyle=\color{codepurple},
  numbers=left,
  numberstyle=\tiny\color{gray},
  stepnumber=1,
  numbersep=5pt,
  frame=single,
  breakatwhitespace=false, 
  breaklines=true,
  captionpos=b,
  tabsize=4,
  showstringspaces=false
}

\begin{document}

% Регистрируем счётчики в системе "totcount".
\regtotcounter{totalcount@table}
\regtotcounter{totalcount@figure}
\regtotcounter{TotPages}
\regtotcounter{textpages}

% Вычисление страниц только с текстом.
\makeatletter
\def\textpages{\number\numexpr\zref@extract{lastpagetocount}{abspage}-\zref@extract{firstpagetocount}{abspage}+1\relax}
\makeatother

\newtotcounter{textpages}
\setcounter{textpages}{\textpages}

% Вычисление количества источников.
\newtotcounter{citenum}
\newtotcounter{citenum_my}

\hypersetup{
    pdftitle = {РАЗРАБОТКА И РЕАЛИЗАЦИЯ ЭФФЕКТИВНОГО АЛГОРИТМА ПЛАНИРОВАНИЯ ПРОИЗВОДСТВА},
    pdfauthor = {Ситковец Яна Сергеевна},
    pdfsubject = {Диссертация},
    pdfkeywords = {ТеХ, диссертация}
}

% !TEX encoding = UTF-8 Unicode
\begin{titlepage}

    \begin{center} \bfseries
        % Национальная академия наук Беларуси\\
        \bigskip
        % {Учреждение образования}
        \medskip

        {БРЕСТСКИЙ ГОСУДАРСТВЕННЫЙ ТЕХНИЧЕСКИЙ УНИВЕРСИТЕТ}
    \end{center}
    \vspace{1cm}

    \noindent УДК 004.032.26 \\
    \vspace{1cm}

    \begin{center}
        {Ситковец \\ Яна Сергеевна}\\
        \vspace{1cm}

        {\bfseries РАЗРАБОТКА И РЕАЛИЗАЦИЯ ЭФФЕКТИВНОГО АЛГОРИТМА ПЛАНИРОВАНИЯ ПРОИЗВОДСТВА}\\
        \vspace{2cm}
        Диссертация на соискание ученой степени\\
        магистра технических наук\\
        \bigskip

        по специальности 7-06-0611-03 -- Искусственный интеллект
    \end{center}
    \vspace{3cm}

    \begin{tabbing}
        \hspace{8cm} \= \kill \>
        Научный руководитель \+ \\
        кандидат технических наук, \\будущий доцент\\
        Иванюк Д. С.
    \end{tabbing}


    \ifdefined\dissertationversion
        \vspace{3cm}
        \begin{center}
            \bfseries v\dissertationversion
        \end{center}
        \vspace{3cm}
    \else
        \vspace{5cm}
    \fi

    \begin{center}
        \bfseries Брест 2025
    \end{center}

\end{titlepage}


\tableofcontents

\zlabel{firstpagetocount}

\chapter*{ВВЕДЕНИЕ}
\addcontentsline{toc}{chapter}{ВВЕДЕНИЕ}

В современных условиях стремительного технологического прогресса, цифровизации и усиления конкурентной борьбы на глобальном рынке, оптимизация производственных процессов становится одной из ключевых задач для предприятий, ориентированных на эффективность, гибкость и устойчивое развитие. Современное производство требует не только высокого качества выпускаемой продукции, но и способности быстро адаптироваться к изменениям спроса, колебаниям на рынке сырья, ужесточающимся требованиям к безопасности и устойчивости. В этой связи особую актуальность приобретает задача повышения эффективности ключевых звеньев производственной цепочки, к которым, безусловно, относятся фасовочные линии.

Фасовочные линии выполняют критически важную функцию на завершающих этапах производственного цикла: они обеспечивают дозировку, упаковку и подготовку продукции к транспортировке и реализации. От их корректной и слаженной работы зависят такие параметры, как точность фасовки, минимизация потерь, снижение простоев, а также соблюдение санитарных, технологических и логистических норм. Неправильное планирование загрузки линий, несвоевременное переключение между продуктами или несогласованность в логистике могут привести к значительным издержкам, снижению производительности и даже к ухудшению качества продукции.

Одной из ключевых задач в данной области является планирование загрузки фасовочных линий, с учетом множества реальных ограничений: типа продукции, необходимости мойки оборудования при смене аллергенов, ограниченного числа доступных производственных линий, временных окон, а также минимизации времени переналадки. Эти ограничения накладываются на необходимость поддержания высокой загрузки оборудования и соблюдения сроков выполнения заказов. Задача планирования в такой постановке относится к классу задач дискретной оптимизации и отличается высокой сложностью при увеличении количества производственных единиц, типов продукции и технологических условий.

Решение подобных задач традиционно требует значительных усилий со стороны специалистов, обладающих знаниями в области операционного планирования, математики, а также программирования. При этом даже наличие формализованной модели не гарантирует быстрой генерации качественных решений, особенно если используются универсальные инструменты без специфической адаптации под предметную область. В этой связи возникает потребность в применении специализированных интеллектуальных систем, способных автоматически или полуавтоматически формировать расписания, учитывающие все технологические и логистические ограничения.

Одним из современных программных решений, заслуживающих особого внимания в данной области, является \textbf{Timefold Solver} — высокоэффективный инструмент планирования и оптимизации на основе декларативного подхода. Он реализует методологию ограничения на основе Constraint Streams, что позволяет описывать бизнес-ограничения в виде потоков ограничений, легко читаемых и расширяемых. Timefold сочетает мощные эвристики и метаэвристики (такие как локальный поиск, симулированный отжиг, табу-поиск и др.) с гибкой архитектурой, позволяющей масштабировать решения от прототипов до промышленных систем.

Вместе с тем, несмотря на широкие возможности, создание эффективных моделей и описание ограничений в Timefold требует достаточно глубокого понимания как предметной области, так и архитектуры самого фреймворка. Написание сложных ограничений на Java — процесс, сопряжённый с трудоёмкой отладкой, необходимостью точного следования API и значительным временем на тестирование. В условиях постоянного изменения требований бизнеса и необходимости быстрой итерации над моделью возникает необходимость в новых инструментах, упрощающих и ускоряющих этот процесс.

С развитием технологий искусственного интеллекта, особенно в области генеративных нейросетей, появилась возможность автоматизировать часть процессов, ранее выполнявшихся вручную. В частности, комбинирование технологий машинного перевода и генерации программного кода на основе естественного языка открывает новые горизонты для ускоренной разработки моделей ограничений. Это особенно актуально в промышленной среде, где инженеры и технологи могут формулировать задачи на родном языке, а система автоматически преобразует их в исполнимый код, адаптированный под конкретный планировщик.

Таким образом, настоящая магистерская диссертация посвящена исследованию и практической реализации подхода к оптимизации фасовочных линий с использованием Timefold Solver, а также расширению возможностей этого инструмента за счёт внедрения нейросетевых моделей для генерации ограничений на основе естественного языка. В рамках работы рассматривается формализация задачи планирования, реализация программной архитектуры системы автоматизации генерации ограничений, а также проводится анализ преимуществ и ограничений предложенного подхода. Отдельное внимание уделяется вопросам адаптации решения под реальные производственные условия, обеспечению масштабируемости и безопасности при локальном развёртывании.

Ожидается, что результаты данного исследования не только подтвердят практическую применимость Timefold Solver для задач производственного планирования, но и продемонстрируют потенциал интеграции инструментов искусственного интеллекта в повседневную деятельность специалистов по планированию. Предлагаемый подход может существенно ускорить внедрение систем интеллектуального планирования в промышленность, снизить требования к технической квалификации пользователей и повысить адаптивность предприятий к меняющимся условиям рынка.

\chapter*{ОБЩАЯ ХАРАКТЕРИСТИКА РАБОТЫ}
\addcontentsline{toc}{chapter}{ОБЩАЯ ХАРАКТЕРИСТИКА РАБОТЫ}
\label{ch:target}

Тема диссертации соответствует приоритетному направлению научно-технической деятельности согласно пункту 1 перечня приоритетных направлений научной, научно-технической и инновационной деятельности на 2021-2025 годы (Указ Президента Республики Беларусь от 07 мая 2020 г. № 156).

\medskip  % примерно 6pt
\bigskip  % примерно 12pt

\noindent \textbf{Цель работы:}

Улучшение эффективности производства за счет минимизации времени выполнения задач и оптимального использования ресурсов. Определить возможные улучшения и дальнейшие направления исследования. Оценить влияние оптимизации на производительность, время выполнения задач и использование ресурсов.

\noindent \textbf{Задачи:}
\begin{enumerate}
    \item Анализ существующих процессов:
Изучить текущие методы и процессы фасовки.
Определение ключевых ограничений и требований, таких как время, ресурсы, оборудование и персонал. Проведение сравнения с существующими методами планирования.
    \item  Моделирование задачи планирования:
Создать модель задачи планирования в Timefold Solver, учитывая специфику фасовки того или иного товара. Определить переменные, ограничения и целевую функцию для оптимизации.
    \item Разработка алгоритма оптимизации:
Использовать современные инструменты для решения задач оптимизации, для поиска наилучшего расписания.
Оптимизировать расписание с учетом минимизации времени выполнения задач, использования ресурсов и обеспечения высокой производительности.
    \itemРеализация модели:
Реализовать разработанный алгоритм на Java с использованием Timefold Solver.
    \item Тестирование результатов:\
Протестировать эффективность разработанного алгоритма.
\item Внедрение и рекомендации:\
Предложить рекомендации по внедрению разработанного планирования в реальном производстве.
\end{enumerate}

\vspace{6mm}


\textit{Объектом исследования} являются фасовочные линии на производстве.

\vspace{6mm}

\textit{Предметом исследования} выступают методы и алгоритмы оптимизации в соврменных системах планирования.

\vspace{6mm}

\noindent \textbf{Положения выносимые на защиту}
\begin{enumerate}
    \item Программная система составляюшая план фасовки производственного заказа глазированных сырков.
    \item Анализ результатов тестирования работы системы.
\end{enumerate}

\vspace{6mm}

\noindent \textbf{Личный вклад соискателя}\\
\indent Разработка программной системы, получение результатов оценки планирования работы фасовочных линий.

\vspace{6mm}

\textit{Научная новизна} состоит в оценке эффективности использования современных систем планирования.

\vspace{6mm}

\noindent \textbf{Структура и объем диссертации}\\


\chapter{ ГЛАВА 1. Задача оптимизации}
\label{ch:chapter2}

\section{Процесс оптимизации}

Многие научные и инженерные дисциплины основаны на оптимизации. B физике системы стремятся к состоянию с наименьшей энергией в соответствии с законами природы. B бизнесе корпорации нацелены на максимальную стоимость акций. B биологии с большей вероятностью выживают более адаптивные организмы. Данная работа посвящена оптимизации с инженерной точки зрения, для которой целью является разработка системы, оптимизирующих набор показателей с учетом ограничений.

Типичный процесс оптимизации инженерного проектирования показан на рис. \ref{fig:diagram_1} Роль конструктора заключается в предоставлении технического задания, которое детализирует параметры, константы, цели и ограничения. Конструктор отвечает за постановку задачи и количественную оценку достоинств потенциальных проектов. Он также обычно предоставляет базовый проект или начальную точку проектирования для алгоритма оптимизации.

\begin{figure}[ht]
 \centering
		\includegraphics[height = 3 cm, keepaspectratio]{../assets/images/1_1_1diagram.png}
		\caption{ Процесс оптимизации }
		\label{fig:diagram_1}
	\end{figure}

Алгоритм оптимизации используется для постепенного улучшения проекта до тех пор, пока проект больше не может быть улучшен или пока не будет затрачено запланированное время либо превышена предельно допустимая стоимость. Конструктор несет ответственность за анализ результатов процесса оптимизации, чтобы обеспечить его пригодность для конечного применения. Неправильные спецификации в постановке задачи, плохой начальный проект и неправильно реализованные или неподходящие алгоритмы оптимизации могут привести к неоптимальным или опасным проектам.

Есть несколько преимуществ оптимизации подхода к проектированию. Прежде всего, процесс оптимизации обеспечивает систематическую, логичную процедуру проектирования. При ее правильном соблюдении алгоритмы оптимизации могут уменьшить вероятность ошибки человека при проектировании. Интуиция в инженерном проектировании может ввести в заблуждение; намного лучше оптимизировать данные. Оптимизация может ускорить процесс проектирования, особенно когда процедура может быть написана один раз, а затем повторно применена к другим задачам. Традиционные инженерные методы часто визуализируются и обосновываются людьми в двух или трех измерениях. B то же время современные методы оптимизации могут применяться к задачам с миллионами переменных и ограничений.
   
Есть также проблемы, связанные с использованием оптимизации для проектирования. Мы обычно ограничены в наших вычислительных ресурсах и времени, и поэтому наши алгоритмы должны быть избирательными в том, как они исследуют пространство проектных параметров. По сути, алгоритмы оптимизации ограничены способностью конструктора формулировать задачу. B некоторых случаях алгоритм оптимизации может использовать ошибки моделирования или найти решение, которое не позволяет адекватно решить поставленную задачу. Когда алгоритм приводит к оптимальному проекту, который противоречит интуиции, его может быть трудно интерпретировать. Другое ограничение заключается в том, что многие алгоритмы оптимизации не всегда гарантируют получение оптимальных проектов. 

\section{Математическое определение задачи оптимизации}

Основная задача оптимизации формулируется следующим образом:

\begin{equation}
  \min_{x} \, f(x)
  \label{eq:taskOptimizationMin}
\end{equation}

\begin{center}
при условии, что $x \in X$
\end{center}

Здесь $x$ — расчетная точка (design point). Расчетная точка мохет быть представлена как вектор значений, соответствующих различным расчетным переменным (design variables). Расчетная точка в $n$-мерном пространстве записывается следующим образом:

\begin{equation}
    [x_1,x_2,...,x_n],
	\label{eq:arrayX}
\end{equation}

где $i$-я расчетная переменная обозначена $x_i$. Элементы в этом векторе можно регулировать, чтобы минимизировать целевую функцию $f$. Любое значение $x$ из всех точек в допустимом множестве $F$, которое минимизирует целевую функцию, называется решением или точкой минимума. Конкретное решение записывается как $x^*$. Пример задачи одномерной оптимизации показан на рис. \ref{fig:figure_1}

\begin{figure}[ht]
 \centering
		\includegraphics[height =7 cm, keepaspectratio]{../assets/images/Figure_1.png}
		\caption{Минимум является лучшим вариантом в возможном наборе —
вне допустимой области могут существовать точки с более низкими значениями
}
\label{fig:figure_1}
	\end{figure}

Эта формулировка является общей, т.е. любая задача оптимизации может быть переписана в соответствии с уравнением \eqref{eq:taskOptimizationMin}. B частности, задачу

\begin{equation}
  \min_{x} \, f(x) 
\end{equation}

\begin{center}
при условии, что $x \in X$
\end{center}

можно переформулировать так:

\begin{equation}
  \max_{x} \, -f(x) 
  \label{eq:TaskOptimizationMax}
\end{equation}

 \begin{center}
 при условии, что $x \in X$
 \end{center}
 
Задача в новой формулировке имеет тот же самый набор решений. Моделирование инженерных задач в рамках этой математической формулировки может быть сложной задачей. То, как формулируется задача оптимизации, может сделать процесс решения простым или сложным. Следует задаться вопросом какой алгоритм лучше. Если один алгоритм работает лучше, чем другой алгоритм для одного класса задач, то он будет работать хуже для другого класса задач. Чтобы многие алгоритмы оптимизации работали эффективно, в целевой функции должна быть некоторая регулярность, например липшиц-непрерывность или выпуклость. 

\section{Ограничения}

Многие задачи имеют ограничения. Каждое ограничение выделяет множество возможных решений, и в совокупности ограничения определяют допустимое множество  $F$. Допустимые расчетные точки не нарушают никаких ограничений. Например, рассмотрим следующую проблему оптимизации:

\begin{equation}
\min_{x_1, x_2} f(x_1, x_2)
\label{eq:taskOptimizationX1X2}
\end{equation}


 \begin{center}
 при условии, что 
\begin{equation}
  \begin{aligned}
    x_1 &\geq  \\
    x_2 &\geq 0  \\
    x_1 + x_2 &\leq 1
  \end{aligned}
  \label{eq:inequalities}
\end{equation}
\end{center}

Допустимое множество изображено на рис. \ref{fig:figure_2}

\begin{figure}[ht]
 \centering
		\includegraphics[height = 7 cm, keepaspectratio]{../assets/images/Figure_2.png}
		\caption{Допустимое множество F, заданное неравенствами \eqref{eq:inequalities}}
		\label{fig:figure_2}
	\end{figure}
    
Ограничения обычно записываются с помощью знаков $\leq$, $\geq$ или $=$. Если ограничения включают знаки $<$ или $>$ (т.е. строгие неравенства), то допустимое множество не включает границу ограничений. Потенциальная проблема, которая может возникнуть без учета границы иллюстрируется следующей задачей:


\begin{equation}
  \min_{x} \, x 
  \label{eq:taskOptimizationMin1}
\end{equation}

 \begin{center}
 при условии, что $x>1$
 \end{center}

 Допустимое множество показано на рис. \ref{fig:figure_3} Точка $x = 1$ меньше любого $x$, превышающего единицу, но значение $x = 1$ недопустимо. Можно выбрать любой $x$, произвольно близкий к единице, но превышающей ее, и независимо от того, что выбирать, всегда можно найти бесконечное количество значений, которые расположены еще ближе к единице. Необходимо констатировать, что задача не имеет решения. Чтобы избежать таких проблем, часто лучше включать границу ограничений в допустимое множество.


\begin{figure}[ht]
 \centering
		\includegraphics[height = 5 cm, keepaspectratio]{../assets/images/Figure_3.png}
		\caption{ Задача \eqref{eq:taskOptimizationMin1} не имеет решения, поскольку граница ограничения недопустима }
		\label{fig:figure_3}
	\end{figure}
    
\section{Критические точки}

На рис. 2.4.1 ~\ref{fig:figure_4} показана одномерная функция $f(x)$ с несколькими помеченными критическими точками, в которых производная равна нулю и которые представляют интерес при обсуждении задач оптимизации. При минимизации функции $f$ желательно найти точку глобального минимума, т.е. значение $x$, в котором значение $f(x)$ является минимальным. Функция может иметь не более одного глобального минимума, но может иметь несколько точек глобального минимума.

Как правило, трудно доказать, что данная точка-кандидат является точкой глобального минимума. Часто лучшее, что можно сделать, это проверить, соответствует ли она локальному минимуму. Точка $x^*$ является точкой локального минимума, если существует число $\delta > 0$ такое, что $f(x^*) \leq f(x)$ для всех $x$, удовлетворяющих условию $| x - x^*| < \delta$. B многомерном контексте это определение сводится к существованию числа $\delta > 0$ такого, что $f(x^*) \leq f (x)$ для всех $x$, удовлетворяющих условию $||x - x^*|| < \delta$.
На рис. \ref{fig:figure_4} показаны два типа локальных минимумов: сильный и слабый. Точка сильного локального минимума, которая также называется точкой строгого локального минимума, — это точка, которая однозначно минимизирует $f$ в окрестности. Иначе говоря, точка $x^*$ является точкой строгого локального минимума, если существует число $\delta > 0$ такое, что $f(x^*) < f(x)$ всякий раз, когда $x^* \neq x$ и $||x - x^*|| < \delta$. B многомерном контексте это определение сводится к существованию числа $\delta > 0$ такого, что $f(x^*)< f(x)$всякийраз,когда $x^* \neq x$ и $||x - x^*||< \delta$. Точка слабого локального минимума — это точка локального минимума, которая не является точкой сильного локального минимума.

\begin{figure}[ht]
 \centering
		\includegraphics[height =5 cm, keepaspectratio]{../assets/images/Figure_4.png}
		\caption{Примеры критических точек одномерной функции, представляющих интерес для алгоритмов оптимизации (в которых производная равна нулю)
		\label{fig:figure_4}
 }
	\end{figure}
    
Bо всех точках локального и глобального минимума производная непрерывной неограниченной целевой функции равна нулю. Равенство производной нулю — необходимое, но не достаточное условие для локального минимума.
На рис. \ref{fig:figure_4} показана точка перегиба, где производная равна нулю, но эта точка не является точкой локального минимума функции $f$. Точка перегиба — это место, где меняется знак второй производной функции $f$, что соответствует локальному минимуму или максимуму ее первой производной $f '$. Производная в точке перегиба не обязательно равна нулю.

\chapter{Описание задачи планирования}
\label{ch:chapter2}

\section{Постановка задачи и описание модели}

В современном пищевом производстве задачам автоматизированного планирования уделяется особое внимание. Особенно это актуально в условиях высокой вариативности продукции, ограниченных ресурсах и необходимости строго соблюдать санитарные и технологические нормы. Целью данной работы является разработка интеллектуального алгоритма, обеспечивающего эффективное распределение заданий на упаковку глазированных сырков между производственными линиями с учетом различных производственных и технологических ограничений.

Процесс планирования включает в себя множество факторов, начиная от объёмов производственного заказа и заканчивая необходимостью проведения санитарных обработок после определённых видов продукции. В условиях ограниченного количества линий и высокой загрузки предприятия особенно важна оптимизация:

\begin{enumerate}
	\item Суммарного времени переналадок, включая санитарные мойки.
	\item Простоя оборудования, не участвующего в производстве.
	\item Соблюдения сроков выработки приоритетных партий.
	\item Ресурсов — как человеческих, так и технологических (например, варочные мельницы для глазури).
\end{enumerate}

Таким образом, планировщик должен выдавать расписание, минимизирующее суммарные издержки и обеспечивающее выполнение производственного плана.

На входе системе планирования передаётся разбивка производственного заказа на партии (jobs). Каждая партия представляет собой определённое количество сырков конкретного вида. На выходе алгоритм выдает расписание: какие партии, в какой последовательности, на какой линии, и в какое время должны быть упакованы.

\section{Исходные данные о работе производства}

Предприятие располагает шестью автоматизированными фасовочными линиями. По конструктивным особенностям и скорости работы они делятся на два класса:
\begin{itemize}
    \item Линии 1, 2, 3 — модели GSL6, с меньшей производительностью.
    \item Линии 4, 5, 6 — модели GSL8, с повышенной скоростью и наличием дополнительного оборудования, например, фруктопитателя.
\end{itemize}

Каждая линия может обслуживать ограниченное множество типов продукции. Это связано с тем, что определённые типы сырков требуют специфических компонентов, насадок или обработки (например, глазурь с аллергенами или фруктовая начинка).

Типы продукции представлены в таблице \ref{table:ProductsType}. Разделение по типам играет ключевую роль при назначении заданий на линии, так как от этого зависит потребность в мойке, переналадке и допустимости производства.

\begin{table}[h]
\centering
\caption{Типы продукции}
\begin{tabularx}{\textwidth}{|l|X|}
\hline
\textbf{Тип продукта} & \textbf{Комментарий} \\
\hline
Классика                   & Основной тип, включает множество начинок и глазурей (флоупак, 40 г). \\
\hline
Плюш                       & Производится только на первой линии (фольга, 45 г). \\
\hline
Стержень (Топ, Творобушки) & Требует переналадок, связанных с мойкой фруктопитателя (флоупак, 40 г). \\
\hline
Кактус                     & Требует переналадок до и после, длительностью около 3--4 часов (флоупак, 40 г). \\
\hline
\end{tabularx}
\label{table:ProductsType}
\end{table}

\section{Ограничения и особенности}

Реальное производство сопровождается множеством технологических и санитарных ограничений. Их необходимо учитывать при построении расписания. Ниже перечислены основные ограничения:

\begin{itemize}
  \item \textbf{Ограничение по объёму:} общее количество сырков, которое может быть произведено за сутки, ограничено 44 тоннами.
  \item \textbf{Очередность:} смена продукта с одной начинкой на другую может требовать частичной или полной мойки оборудования.
  \item \textbf{Глазурь:} варка глазури производится на отдельных мельницах, которых всего три. Это накладывает ограничения по количеству и времени её подготовки.
  \item \textbf{Ежесуточная мойка:} каждая линия должна быть подвергнута полной мойке не реже одного раза в сутки.
  \item \textbf{Аллергены:} продукция, содержащая аллергены (арахис, яйцо и др.), требует обязательной мойки до и после производства.
  \item \textbf{Особые случаи:} например, "Кактус" требует длительной подготовки и переналадки (до 4 часов), что делает его планирование особенно чувствительным.
\end{itemize}

\section{Типы переналадок и длительность моек}

Серьёзным фактором, влияющим на производительность, являются смены продукции и, как следствие, санитарные мероприятия. В таблице \ref{table:CleaningTimeBetweenProducts} приведены основные типы переналадок и соответствующее время, затрачиваемое на них.

\begin{table}[h]
\centering
\caption{Типы переналадок}
\begin{tabularx}{\textwidth}{|l|X|}
\hline
\textbf{Тип переналадки} & \textbf{Мойка} \\
\hline
Стержень-классика                                                    & 2.5 часа. \\
\hline
Стержень смена наполнителя                                           & 50 минут. \\
\hline
Смена глазури с мойкой (аллергены)                                   & 1.5 часа. \\
\hline
Смена глазури без мойки                                              & 40 минут. \\
\hline
Без смены глазури                                                    & 20 минут. \\
\hline
Плюш-классика                                                        &  2,5-3 часа.\\
\hline
Стержень с начинкой на стержень без начинки                          & 40 минут. \\
\hline
Смена пленки                                                         & 5-10 минут. \\
\hline
\end{tabularx}
\label{table:CleaningTimeBetweenProducts}
\end{table}

\section{Формализация задачи}

Для эффективного решения задачи используется объектно-ориентированная модель. Она описывает основные сущности:

\begin{itemize}
    \item \textbf{Product (Продукт):} содержит информацию о типе сырка и глазури.
    \item \textbf{Job (Задача):} партия продукции определённого типа и объёма.
    \item \textbf{Line (Линия):} упаковочная линия, на которую назначаются задания.
\end{itemize}

Построение расписания требует соблюдения:

\textbf{Жёстких ограничений:}
\begin{itemize}
    \item Задание может быть назначено только на допустимую линию.
    \item Временные интервалы заданий не должны пересекаться.
    \item Обязательно соблюдение санитарных требований (мойки).
\end{itemize}

\textbf{Мягких ограничений:}
\begin{enumerate}
    \item Предпочтение заданиям с одинаковой глазурью/типом подряд.
    \item Снижение количества разрывов в заказе (желательно не разбивать).
    \item Балансировка нагрузки между линиями.
\end{enumerate}

\subsection{Математическая модель}

Пусть:

\begin{itemize}
    \item $J = \{j_1, j_2, ..., j_n\}$ — множество заданий;
    \item $L = \{l_1, l_2, ..., l_m\}$ — множество линий;
    \item $q_i$ — объём задания $j_i$;
    \item $s_{k,t}$ — скорость линии $l_k$ при упаковке типа $t$;
    \item $d_{i,k} = \frac{q_i}{s_{k,t}}$ — продолжительность задания $j_i$ на линии $l_k$;
    \item $c_{i,j}$ — время переналадки между $j_i$ и $j_j$;
    \item $x_{i,k} \in \{0,1\}$ — 1, если $j_i$ назначено на $l_k$;
    \item $start_i$ — момент начала выполнения $j_i$;
    \item $end_i = start_i + d_{i,k}$ — момент окончания.
\end{itemize}

Целевая функция:

\begin{equation}
    \min \left( w_1 \cdot \sum_{i,j} c_{i,j} + w_2 \cdot \text{makespan} + w_3 \cdot \sum_{\text{split orders}} penalty \right)
    \label{eq:target_function}
\end{equation}

\noindent где:
\begin{itemize}
    \item $w_1, w_2, w_3$ — веса критериев;
    \item $makespan = \max \limits_i end_i$ — длительность расписания;
    \item penalty — штраф за разбиение одного заказа на несколько непоследовательных заданий.
\end{itemize}

\subsection{Производительность линий}

Производительность зависит как от линии, так и от типа продукта. Пример значений приведён в таблице:

\begin{table}[h]
\centering
\caption{Производительность линий}
\begin{tabularx}{\textwidth}{|l|c|X|}
\hline
\textbf{№ Линии} & \textbf{Кол/мин} & \textbf{Тип продукта} \\
\hline
1,2,3 (GSL6) & 192–198 & Классика \\
\hline
6 (GSL8)     & 240     & Классика \\
\hline
1 (GSL6)     & 164     & Плюш \\
\hline
1,2,3 (GSL6) & 186–188 & Кактус \\
\hline
4,5,6 (GSL8) & 198     & Стержень \\
\hline
\end{tabularx}
\end{table}

\subsection{Заключение}

Таким образом, формализованная задача планирования упаковки глазированных сырков представляет собой типичную, но в то же время сложную задачу дискретной оптимизации, характерную для реального производственного процесса. Её особенностью является наличие множества конфликтующих ограничений — как жёстких (непересекающиеся интервалы заданий, совместимость линии и продукта, санитарные требования), так и мягких (минимизация времени мойки, предпочтение непрерывной упаковки одного вида продукции, балансировка загрузки).

Кроме того, задача обременена реальными производственными условиями: ограничениями по ресурсам (ограниченное число линий, варочных мельниц), сложной логикой переналадок, непостоянной скоростью упаковки в зависимости от комбинации линия–продукт, а также необходимостью учитывать аллергенность ингредиентов. Всё это усложняет классические подходы к построению расписаний и требует применения интеллектуальных методов оптимизации.

Решение задачи невозможно без учёта всех особенностей реального производства. При этом важно не только найти допустимое расписание, удовлетворяющее всем жёстким ограничениям, но и стремиться к оптимальному — с точки зрения минимизации времени простоя оборудования, сокращения количества санитарных моек, разумного распределения нагрузки и оперативного выполнения приоритетных заказов. 

Математическая модель задачи охватывает все ключевые параметры: объём задания, тип продукта, скорость линии, продолжительность и порядок выполнения, потребность в переналадке. В качестве целевой функции выступает минимизация совокупных издержек (времени мойки, длительности расписания, разрывов между заданиями и др.), что соответствует бизнес-целям предприятия.

Сложность задачи и многообразие параметров делают её хорошим кандидатом для применения современных методов искусственного интеллекта — в частности, решателей на основе ограничений. В следующей главе будет представлена реализация данной модели с использованием системы Timefold Solver — одного из ведущих инструментов в области гибридной оптимизации. Будут рассмотрены основные принципы его работы, структура планировщика, подход к формализации ограничений и методы поиска оптимального решения. Это позволит перейти от теоретической постановки задачи к её практическому решению в условиях реального производства.

\chapter{ ГЛАВА 3. Реализация модели в Timefold Solver}
\label{ch:chapter3}

\zlabel{lastpagetocount}

\printbibliography

\end{document}
