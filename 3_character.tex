\chapter*{ОБЩАЯ ХАРАКТЕРИСТИКА РАБОТЫ}
\addcontentsline{toc}{chapter}{ОБЩАЯ ХАРАКТЕРИСТИКА РАБОТЫ}
\label{ch:target}

Тема диссертации соответствует приоритетному направлению научно-технической деятельности согласно пункту 1 перечня приоритетных направлений научной, научно-технической и инновационной деятельности на 2021-2025 годы (Указ Президента Республики Беларусь от 07 мая 2020 г. № 156).

\vspace{3mm}
\aim
\vspace{3mm}

\noindent \textbf{Цель работы:} \\
\indent Улучшение эффективности производства за счет минимизации времени выполнения задач и оптимального использования ресурсов. Определить возможные улучшения и дальнейшие направления исследования. Оценить влияние оптимизации на производительность, время выполнения задач и использование ресурсов.  \\

\noindent \textbf{Задачи:}
\begin{enumerate}
    \item Анализ существующих процессов:
Изучить текущие методы и процессы фасовки.
Определение ключевых ограничений и требований, таких как время, ресурсы, оборудование и персонал. Проведение сравнения с существующими методами планирования.
    \item  Моделирование задачи планирования:
Создать модель задачи планирования в Timefold Solver, учитывая специфику фасовки того или иного товара. Определить переменные, ограничения и целевую функцию для оптимизации.
    \item Разработка алгоритма оптимизации:
Использовать современные инструменты для решения задач оптимизации, для поиска наилучшего расписания.
Оптимизировать расписание с учетом минимизации времени выполнения задач, использования ресурсов и обеспечения высокой производительности.
    \itemРеализация модели:
Реализовать разработанный алгоритм на Java с использованием Timefold Solver.
    \item Тестирование результатов:\
Протестировать эффективность разработанного алгоритма.
\item Внедрение и рекомендации:\
Предложить рекомендации по внедрению разработанного планирования в реальном производстве.
\end{enumerate}

\vspace{3mm}
\novelty
\vspace{3mm}

\textit{Объектом исследования} являются фасовочные линии на производстве.

\vspace{3mm}
\novelty
\vspace{3mm}

\textit{Предметом исследования} выступают методы и алгоритмы оптимизации в соврменных системах планирования.

\vspace{3mm}
\novelty
\vspace{3mm}

\noindent \textbf{Положения выносимые на защиту}
\begin{enumerate}
    \item Программная система составляюшая план фасовки производственного заказа глазированных сырков.
    \item Анализ результатов тестирования работы системы.
\end{enumerate}

\vspace{3mm}
\novelty
\vspace{3mm}

\noindent \textbf{Личный вклад соискателя}\\
\indent Разработка программной системы, получение результатов оценки планирования работы фасовочных линий.

\vspace{3mm}
\novelty
\vspace{3mm}

\textit{Научная новизна} состоит в оценке эффективности использования современных систем планирования.

\vspace{3mm}
\novelty
\vspace{3mm}

\noindent \textbf{Структура и объем диссертации}\\
